\chapter{Conclusion}
The main goals of this thesis were to create monitoring tool with small footprint on the monitored application and high-level application transparency and tool universality. It was also desired to ensure the usage and deployment of the final tool is simple. 

The instrumentation overhead can be still observed even though we are instrumenting the classes in a separated instrumentation machine, however it is just a constant overhead based on the nature of injected code to the classes. The tool universality was achieved by implementing the native agent universal to all Java applications and by creating a core instrumentation server. This core server can be extended by developers and they can create an application specific instrumentation tools. This also ensures that the final users of the monitored application does not need to know about the monitoring and can work with the application as usually. The developer has also possibility to extend the monitoring platform by custom user interface and also can specify custom format for spans being exported from the application. This ensures that the Distrace tool may be integrated easily to already existing environments. The interface to be extended at the core instrumentation server is kept simple in order to make the usage simple also for developers.

Comparing to related Google Dapper, the tool introduced by this thesis is open-source and allows higher application transparency since purpose of Google Dapper is to monitor only Google applications. Comparing to Zipkin, the user does not need to change the application sources in order to attach span and trace information. This ensures that the source code of the original application remains unchanged. The tool provided by this thesis does not aim to replace any of the mentioned tools, however it tries to create universal tool with keeping performance in mind and ensuring the usage for the end-user is as simple as possible.

\section{Future plans}
 The final section points out the future plans related to the Distrace tool.
This tool is planned to be extend in the future. Mainly, the tool should be improved in the following areas:
\begin{itemize}
	\item \textbf{More Additional Span Exporters} \newline
	Currently, the tool provides two default span exporters and allows the user to extend the \texttt{SpanExporter} abstract class and implement custom ones. However, we would like to create more exporters in the future, which would be able to store spans into different storage types and also in different formats. At this moment, the output is in the JSON format understandable to the Zipkin user interface and the data are exported either to disk or are send to the user interface right away. We could, for example, create a span exporter, which could export spans into a database. from which the arbitrary user interface could fetch the data.
	\item \textbf{Support for Flame Graphs} \newline
	The second future plan is to add support for flame charts. The native agent could be used to capture the stack-traces of the running application and later, a flame graph representing the distributed computation could be created. For example, this integration would give us the ability to inspect the memory-usage or performance cluster-vise using the flame graphs visualization.
\end{itemize}

