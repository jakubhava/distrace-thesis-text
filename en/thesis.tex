%%% The main file. It contains definitions of basic parameters and includes all other parts.

%% Settings for single-side (simplex) printing
% Margins: left 40mm, right 25mm, top and bottom 25mm
% (but beware, LaTeX adds 1in implicitly)
\documentclass[12pt,a4paper]{report}
\setlength\textwidth{145mm}
\setlength\textheight{247mm}
\setlength\oddsidemargin{15mm}
\setlength\evensidemargin{15mm}
\setlength\topmargin{0mm}
\setlength\headsep{0mm}
\setlength\headheight{0mm}
% \openright makes the following text appear on a right-hand page
\let\openright=\clearpage

%% Settings for two-sided (duplex) printing
% \documentclass[12pt,a4paper,twoside,openright]{report}
% \setlength\textwidth{145mm}
% \setlength\textheight{247mm}
% \setlength\oddsidemargin{14.2mm}
% \setlength\evensidemargin{0mm}
% \setlength\topmargin{0mm}
% \setlength\headsep{0mm}
% \setlength\headheight{0mm}
% \let\openright=\cleardoublepage

%% Character encoding: usually latin2, cp1250 or utf8:
\usepackage[utf8]{inputenc}

%% Further useful packages (included in most LaTeX distributions)
\usepackage{amsmath}        % extensions for typesetting of math
\usepackage{amsfonts}       % math fonts
\usepackage{amsthm}         % theorems, definitions, etc.
\usepackage{bbding}         % various symbols (squares, asterisks, scissors, ...)
\usepackage{bm}             % boldface symbols (\bm)
\usepackage{graphicx}       % embedding of pictures
\usepackage{fancyvrb}       % improved verbatim environment
\usepackage{natbib}         % citation style AUTHOR (YEAR), or AUTHOR [NUMBER]
\usepackage[nottoc]{tocbibind} % makes sure that bibliography and the lists
			    % of figures/tables are included in the table
			    % of contents
\usepackage{dcolumn}        % improved alignment of table columns
\usepackage{booktabs}       % improved horizontal lines in tables
\usepackage{paralist}       % improved enumerate and itemize
\usepackage[usenames]{xcolor}  % typesetting in color

%%% Basic information on the thesis

% Thesis title in English (exactly as in the formal assignment)
\def\ThesisTitle{Monitoring Tool for Distributed Java Applications}

% Author of the thesis
\def\ThesisAuthor{Jakub Háva}

% Year when the thesis is submitted
\def\YearSubmitted{2017}

% Name of the department or institute, where the work was officially assigned
% (according to the Organizational Structure of MFF UK in English,
% or a full name of a department outside MFF)
\def\Department{Department of Distributed and Dependable Systems}

% Is it a department (katedra), or an institute (ústav)?
\def\DeptType{Department}

% Thesis supervisor: name, surname and titles
\def\Supervisor{Mgr. Pavel Parízek, Ph.D.}

% Supervisor's department (again according to Organizational structure of MFF)
\def\SupervisorsDepartment{Department of Distributed and Dependable Systems}

% Study programme and specialization
\def\StudyProgramme{Computer Science}
\def\StudyBranch{Software Systems}

% An optional dedication: you can thank whomever you wish (your supervisor,
% consultant, a person who lent the software, etc.)
\def\Dedication{%
TODO: Write dedication
}

% Abstract (recommended length around 80-200 words; this is not a copy of your thesis assignment!)
\def\Abstract{%
TODO: Abstract
}

% 3 to 5 keywords (recommended), each enclosed in curly braces
\def\Keywords{%
{key} {words}
}

%% The hyperref package for clickable links in PDF and also for storing
%% metadata to PDF (including the table of contents).
\usepackage[pdftex,unicode]{hyperref}   % Must follow all other packages
\hypersetup{breaklinks=true}
\hypersetup{pdftitle={\ThesisTitle}}
\hypersetup{pdfauthor={\ThesisAuthor}}
\hypersetup{pdfkeywords=\Keywords}
\hypersetup{urlcolor=blue}

% Definitions of macros (see description inside)
\include{macros}

% Title page and various mandatory informational pages
\begin{document}
\include{title}

%%% A page with automatically generated table of contents of the master thesis

\tableofcontents

%%% Each chapter is kept in a separate file
\include{preface}
\chapter{Project Goals}
\chapter{Similar Work}
\section{Google Dapper}
\section{Zipkin}
\chapter{Analysis}
\section{Instrumentation libraries}
\subsection{Javassist}
\subsection{ByteBuddy}
\subsection{CGlib}
\subsection{ASM}
.. just give brief overview what were the instrumentation libreries choices. The selected one will be describied in the next section
\section{Communication Middleware}
\subsection{ZeroMQ}
\subsection{NanoMSG}

\section{Comparison of Agent Approaches}
\subsection{Java Agent Solution}
\subsection{Native Agent Solution}

\chapter{Used Technologies}
\section{Java}
\subsection{Class Initialization Process}
\subsection{JVMTI}
\subsection{JNI}
\subsection{ClassLoaders}
\section{ByteBuddy}
\subsection{Main Concept}
\subsection{Transformers}
\subsection{Intereptors}
\subsection{Class File Locator}
\subsection{Advice API}
\section{NanoMgs}
\subsection{C++11 Mapping}
\subsection{Java Mapping}

\section{spdlog}
logging library used
\chapter{Platform Architecture}
\section{Architecture Description}
\section{Communication}

\chapter{Native Agent}
\section{Structure Overview}
\section{Instrumentation API}
\section{Byte Class Parsing}
\section{Instrumentation}
\section{Native Agent Arguments}
\chapter{Instrumentation Server}
\section{Instrumentation Handling}
\section{Communication modes}
\section{Class Caching}
\chapter{Instrumentation Library}
\section{Custom Service Loader}
\section{Public interfaces}
\section{Extending the Library}
.. instrumentation server can run on the same node or over the network. Instrumentation server can have client code attached or not.
\section{ClassLoaders}
\section{JSON Generation}
\chapter{User Interface}
\section{Zipkin Overview}
\section{Zipkin Data Model}
\section{Zipkin JSON Format}
\chapter{Collectors}
Should I mention the collectors ? It may be sufficient to have send data right to zipkin for demonstration purposes
\chapter{Deployment Strategies}
\section{Instrumentor on the same node with the Application}
\section{Instrumentor available over the Network}
\section{Bundling the application classes with the Instrumentor}
\chapter{Platform demonstration}
\section{Bulding Monitoring tool on top of Distrace}
\section{Basic Demonstration}
\section{Optimizing the instrumentation}
\chapter{Future plans}
\section{Integration with well-known data collectors}
\section{Add support for Flame charts}
\chapter{Docker Support}
\chapter{Conclusion}
An~example citation: \cite{Andel07}

\section{Title of the first subchapter of the first chapter}

\section{Title of the second subchapter of the first chapter}



\include{epilog}

%%% Bibliography
\include{bibliography}

%%% Figures used in the thesis (consider if this is needed)
\listoffigures

%%% Tables used in the thesis (consider if this is needed)
%%% In mathematical theses, it could be better to move the list of tables to the beginning of the thesis.
\listoftables

%%% Abbreviations used in the thesis, if any, including their explanation
%%% In mathematical theses, it could be better to move the list of abbreviations to the beginning of the thesis.
\chapwithtoc{List of Abbreviations}

%%% Attachments to the master thesis, if any. Each attachment must be
%%% referred to at least once from the text of the thesis. Attachments
%%% are numbered.
%%%
%%% The printed version should preferably contain attachments, which can be
%%% read (additional tables and charts, supplementary text, examples of
%%% program output, etc.). The electronic version is more suited for attachments
%%% which will likely be used in an electronic form rather than read (program
%%% source code, data files, interactive charts, etc.). Electronic attachments
%%% should be uploaded to SIS and optionally also included in the thesis on a~CD/DVD.
\chapwithtoc{Attachments}

\openright
\end{document}
