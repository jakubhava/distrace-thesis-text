%%% The main file. It contains definitions of basic parameters and includes all other parts.

%% Settings for single-side (simplex) printing
% Margins: left 40mm, right 25mm, top and bottom 25mm
% (but beware, LaTeX adds 1in implicitly)
\documentclass[12pt,a4paper]{report}
\setlength\textwidth{145mm}
\setlength\textheight{247mm}
\setlength\oddsidemargin{15mm}
\setlength\evensidemargin{15mm}
\setlength\topmargin{0mm}
\setlength\headsep{0mm}
\setlength\headheight{0mm}
% \openright makes the following text appear on a right-hand page
\let\openright=\clearpage

%% Settings for two-sided (duplex) printing
% \documentclass[12pt,a4paper,twoside,openright]{report}
% \setlength\textwidth{145mm}
% \setlength\textheight{247mm}
% \setlength\oddsidemargin{14.2mm}
% \setlength\evensidemargin{0mm}
% \setlength\topmargin{0mm}
% \setlength\headsep{0mm}
% \setlength\headheight{0mm}
% \let\openright=\cleardoublepage

%% Character encoding: usually latin2, cp1250 or utf8:
\usepackage[utf8]{inputenc}

%% Further useful packages (included in most LaTeX distributions)
\usepackage{amsmath}        % extensions for typesetting of math
\usepackage{amsfonts}       % math fonts
\usepackage{amsthm}         % theorems, definitions, etc.
\usepackage{bbding}         % various symbols (squares, asterisks, scissors, ...)
\usepackage{bm}             % boldface symbols (\bm)
\usepackage{graphicx}       % embedding of pictures
\usepackage{fancyvrb}       % improved verbatim environment
\usepackage[square,numbers]{natbib}         % citation style AUTHOR (YEAR), or AUTHOR [NUMBER]
\usepackage[nottoc]{tocbibind} % makes sure that bibliography and the lists
			    % of figures/tables are included in the table
			    % of contents
\usepackage{dcolumn}        % improved alignment of table columns
\usepackage{booktabs}       % improved horizontal lines in tables
\usepackage{paralist}       % improved enumerate and itemize
\usepackage[usenames]{xcolor}  % typesetting in color
\DeclareGraphicsExtensions{.pdf,.png,.jpg}
\graphicspath{ {../img/} }
\usepackage{listings}
\usepackage{caption}

\usepackage{blindtext}
\lstset{columns=flexible}

%%% Basic information on the thesis

% Thesis title in English (exactly as in the formal assignment)
\def\ThesisTitle{Monitoring Tool for Distributed Java Applications}

% Author of the thesis
\def\ThesisAuthor{Bc. Jakub Háva}

% Year when the thesis is submitted
\def\YearSubmitted{2017}

% Name of the department or institute, where the work was officially assigned
% (according to the Organizational Structure of MFF UK in English,
% or a full name of a department outside MFF)
\def\Department{Department of Distributed and Dependable Systems}

% Is it a department (katedra), or an institute (ústav)?
\def\DeptType{Department}

% Thesis supervisor: name, surname and titles
\def\Supervisor{RNDr. Pavel Parízek, Ph.D}

% Supervisor's department (again according to Organizational structure of MFF)
\def\SupervisorsDepartment{Department of Distributed and Dependable Systems}

% Study programme and specialization
\def\StudyProgramme{Computer Science}
\def\StudyBranch{Software Systems}

% An optional dedication: you can thank whomever you wish (your supervisor,
% consultant, a person who lent the software, etc.)
\def\Dedication{%
I would like to thank my thesis supervisor Dr. Pavel Parízek for leading me throughout the whole thesis and willing to help with any concerns I've ever had. I would also like to thank H2O.ai for being able to write this thesis under their coordination, particularly to Dr. Michal Malohlava for providing very useful technical advice.
}

% Abstract (recommended length around 80-200 words; this is not a copy of your thesis assignment!)
\def\Abstract{%
The main goal of this thesis is to create a monitoring platform and library that can be used to monitor distributed Java-based applications. This work is inspired by Google Dapper and shares a concept called "Span" with it. Spans represents a small specific part of the computation and  are used to capture state among multiple communicating hosts. In order to be able to collect spans without recompiling the original application's code, instrumentation techniques are highly used in the thesis. The monitoring tool, which is called Distrace, consists of two parts: the native agent and instrumentation server. Users of this platform are supposed to extend the instrumentation server and specify the points in their application's code where new spans should be created and closed. In order to achieve high performance and affect the running application at least as possible, the instrumentation server is used for instrumenting the code. The tool is aimed to have a small foot-print on the monitored applications, should be easy to deploy and is transparent to target applications from the point of view of the final user.
}

% 3 to 5 keywords (recommended), each enclosed in curly braces
\def\Keywords{%
{monitoring}, {cluster}, {instrumentation}, {distributed systems}, {performance}
}

%% The hyperref package for clickable links in PDF and also for storing
%% metadata to PDF (including the table of contents).
\usepackage{url}
\def\UrlBreaks{\do\/\do-}


\usepackage[pdftex,unicode]{hyperref}   % Must follow all other packages
\hypersetup{breaklinks=true}
\hypersetup{pdftitle={\ThesisTitle}}
\hypersetup{pdfauthor={\ThesisAuthor}}
\hypersetup{pdfkeywords=\Keywords}
\hypersetup{urlcolor=blue}
% Definitions of macros (see description inside)
\include{macros}
% Title page and various mandatory informational pages
\begin{document}
\include{title}

%%% A page with automatically generated table of contents of the master thesis
\setcounter{tocdepth}{1}
\tableofcontents

%%% Each chapter is kept in a separate file
%%\include{preface}
\chapter{Introduction}
\section{Project Goals}


\chapter{Background}
\label{chap:background}
This chapter covers technologies which are relevant to the thesis. It starts with the overview of similar monitoring tools for cluster based applications and follows by short overview of tools for debugging of large scale applications. Later different approaches to applications profiling are described. 
In the next several sections the technologies which have been consider or are used in the thesis in the current moment are introduced. It covers libraries for bytecode manipulation, communication, logging and Java relevant libraries such as JNI and JVMTI. Docker is briefly described at the end of this chapter as it is used as the main distributed package for the whole platform.

\section{Cluster Monitoring Tools}
The most significant platforms to this thesis are Google Dapper and Zipkin, where Zipkin is based on the previous. Both serves the same core purpose which is to monitor large-scale Java based distributed applications. This thesis is based mainly on Google Dapper but also uses helpful Zipkin modules such as the user interface. Since Zipkin is developed according to Google Dapper design, these two platforms shares very similar concepts. The most important concept is a Span and it is explained in more details in the  following section. For now, we can think of a span as time slots encapsulating several calls from one node to another with well-defined start and end of the communication. The following two sections describes the basics the both mentioned platform. Both Zipkin and Dapper shares very similar concepts wo we just point out the most import parts relevant to the thesis.
\subsection{Google Dapper}
Google Dapper is proprietary software which was mainly developed as a tool for monitoring large distributed applications since debugging and reasoning about applications running on multiple host at the same time, sometimes written in different programming languages is inherently complex. Google Dapper has 3 main pillars on which is built:
\begin{itemize}
	\item Low overhead
	It was assumed that such a tool should share the same life-cycle as the monitored application itself thus low overhead was on of the main design goals as well. Google dapper 
	\item Application level transparency
	The developers and users of the application should not know about the monitoring tool and are not supposed to change the way how they interact with the system. It can be assumed from the paper that achieving application level transparency at Google was easier than it could be in more diverse environments since all the code produced in the Google shares the same libraries and control flow.
	\item Scalability
	Such a system should perform well on large scale data.
\end{itemize}	
Google Dapper collects so called distributed traces. The origin of the distributed trace is the communication/task initiator and the trace spans across the nodes in the cluster which took part as the computation/communication.
	
There were two proposals for obtaining this information - using the black-box and annotation-based monitoring approaches. The first one assumes no additional knowledge about the application whereas the second can use of additional information via annotations. Dapper is mainly using black-box monitoring schema since most of the control flow and RPC subsystems are shared among Google.
	
In Dapper, distributed traces are captured in so called trace trees, where tree nodes are basic units of work referred to as spans. Span is related to other spans via dependency edges. These edges represents relationship between parent span and children of this span. Usually the edges represents some kind of RPC calls or similar kind of communication.

Each span its own id so it can be uniquely identified. In order to reconstruct the whole trace tree, we need to be able to identify the starting Span. Spans without parent id are called root spans serves exactly that purpose. Span can also contain information from multiple hosts, usually from spans from direct neighborhood. Spans structure in Dapper platform is described in the figure \ref{fig:dapper_span}.
\begin{figure}
	\centering
		\includegraphics[scale=0.7]{dapper_span.png}
	\caption{Example of Span in Google Dapper. Picture taken from the Google Dapper paper}
	\ref{fig:dapper_span}
\end{figure}

Dapper is able to achieve application-level transparency and follow distributed control paths thanks to instrumentation of a few common, mostly shared libraries among Google developers. 
\begin{itemize}
	\item Dapper attaches so called trace-context as thread-local variable to the thread when the thread handles any kind of control path. Trace context is small data structure containing mainly just reference to current and parent span via their ids.
	
	\item Dapper instruments the callback mechanism so when computation is deferred, the callbacks still carry around trace context of the creator and therefore also parent span ans current span id
	
	\item Most of the communication in Google is using single RPC framework with language bindings to different languages. This library was instrumented as well to achieve the desired transparency.
\end{itemize}

Even though Dapper is mainly following black-box monitoring scheme mentioned bellow, it still have small support for adding custom annotation to the code. This gives the developer of an application possibility to attach additional information to spans which are very application-specific.

The low-level overhead was also achieved by sampling the data. As is mentioned in the paper, the volume of data at Google is significant so only samples are taken at a time.

\subsection{Zipkin}
Zipkin is open-source distributed tracing system. It based on Google Dapper technical paper and manages both the collection and lookup of captured data.

Zipkin uses instrumentation and annotations for capturing the data. Some
information are captured automatically such as time when Span was created whereas some are optional and some even application-specific.

Zipkin architecture can bee seen on figure \ref{fig:zipkin_architecture}.
\begin{figure}
	\centering
	\includegraphics[scale=0.6]{zipkin_architecture.png}
	\caption{Zipkin architecture - http://zipkin.io/pages/architecture.html}
	\label{fig:zipkin_architecture}
\end{figure}
The instrumented application is responsible for creating valid traces. For that reason Zipkin has set of pre-instrumented libraries ready to be used which works well with whole Zipkin infrastructure. Spans are stored asynchronously in Zipkin to ensure lower overhead.

Once the span is created, it is sent to Zipkin, in more details, to Zipkin collector. In General, Zipkin consists of 4 components:
\begin{itemize}
	\item Zipkin Collector
	It is usually a daemon thread or process which stores, validates and indexes the data for future lockups.
	\item Storage
	Data in zipkin can be stored in a mulltiple ways, so this is a pluggable component. Data can be stored in for example in Cassandra, MySQL or can be send to Zipkin UI right away without storing it anywhere. The last option is good only for small amount of data.
	\item Zipkin Query Service
	This component act as a query daemon allowing us to query various informaion about span using simple JSON API.
	\item Web UI
	Basic, but very useful user interface. The user can see whole trace trees and all spans with dependencies between them.
\end{itemize}
 In the thesis the Zipkin UI is used as front end for developed the monitoring tool and it's format is described in more detail in  \hyperref[sec:zipkin_ui]{Zipkin UI} section of \hyperref[chap:design]{Design} chapter.
 
 The reason why Zipkin UI was selected as the primary user interface for this work is mainly it's simplicity and ease of use. Also it fulfills the visualization requirements of the thesis as well, since we need to see dependencies between spans and also whole trace tree as well. However the monitoring platform is not tightly-coupled with this user interface. We will see later how to create custom span savers which can store data in any format suitable for different visualization tools.
 

\section{Tools for Large-Scale Debugging}
Standard techniques and tools can be used for debugging distributed applications, however when using these tools we lack the information about dependencies between different nodes in the cluster. There are many tools under the category of large-scale debugging but we just point out basic ideas behind two different approaches - discovering scalling bugs and behaviour based debugging. 

\subsection{Discovering Scaling Bugs}
In distributed systems the scalability is very important. It is very important to know how our platform scales when it comes to significantly big data and what is the scalability trend we can expect. It can happen that on large data the platform can run significantly slower than expected when tested on smaller data. We call this issue as a scaling bug. Tools which can be used to help with these kind od bugs are for example Krishna and WuKong. Both of the mention tools are based on the same idea. They build a scaling trend based on data batches of smaller size. The observed scalling trend acts as a boundary. We observe the scalling bug when the scalling trend is violated. In the first tool, Wrishna, we can't tell which port of the program violated the scalling trend, however it is possible in the second tool, WuKong. In comparison to Krishna, Wukong doesn't build one scalling trend of the whole applications, but creates more smaller models, each per some control flow structure in desired programming language where the all these smaller models represent together the whole scalling trend. When we hit into scalling bug, WuKong can give us hints where the trend can be violated.

\subsection{Behavior-based Analysis}
The different category of tools used for debugging of large scale applications are based on behaviour analysis. The basic idea behind these tools is that the classes of equivalence are created from different program processes and different runs. Using this approach we lower down the number of data we need to inspect and the tools can help us to discover anomalies between different observed classes. For example, STAT - Stack Trace Analysis Tool, is a lightweight and scalable debugging tool used for identifying errors on massive high performance computing platforms. It gathers stack traces from all parallel executions, merges together stacktraces from different processes that have the same calling sequence and based on that creates equivalence classes which make it easier for debugging highly parallel applications. As the other example falling under the same category is AutomaDed. This tool creates several models from an execution and can compare them using clustering algorithm with (dis)-similarity metric to to discover anomalous behaviours. It also can point to specific code region which may be causing the anomaly.

\section{Profiling Tools}
Profiling is a form of dynamic code analysis used for analyzing for example how long each part of the system takes in the whole computation, where the computation spends the most time or the memory requirements of the whole program. Generally, we can group the profiling tools into two categories: sampling profilers and instrumentation profilers.

\begin{itemize}
	\item sampling profilers

Sampling profilers take statistical samples of an application at well-defined points such as method invocations. It usually have less overhead comparing to instrumentation profilers. The points were the application should take samples can be inserted at the compilation time by the compiler. Using these profilers we can collect how long the method run, who call it or for example the complete stacktrace. We however can't record any application specific information.
\item instrumentation profilers
This can be solved by instrumentation profilers. These profilers build on the instrumentation of the application's source code. They record the same kind of information as the sampling profilers but usually give us the ability to specify extra points in the code we are interested in and also to record application specific data.
\end{itemize}


However, we can look on profilers from different point of view and categorize them based on the level on which they operate and are able to record the information - system profilers and application specific profilers. 
At application specific profilers, we are the most interested in profilers targeted for JVM platform.
\begin{itemize}
	\item System profilers
	System profilers operate on OS-level. They are great at showing system code paths, but are not able to capture method calls done for example in Java application.
	\item JVM profilers
	These profilers show Java methods, but usually not system code paths.
\end{itemize}
The ideal solution for monitoring purposes would be to have information from both kind of profilers, however combining outputs of these profiler types is not straightforward. The profilers which are able to collect traces from both the profiler types are usually called mixed-mode profilers. JDK8u60 comes with the solution in a way of extra JVM argument \textit{-XX:+PreserveFramePointer} \cite{MixedModeProfilers}.  Operating system is usually using this field to point to most recent call of the stack frame and system profilers make uses of this field. In case of Java, compilers and virtual machines don't need to use this field since they are able to calculate the offset of the latest stack frame from the stack pointer. This leaves this register available for various kind of JVM optimalizations.

This option ensures that JVM abides the frame pointer register and will not use it as general purpose register and therefore we can get both system and JVM stack frames in a one call hierarchy. Using the JVM mixed-mode profilers we are able to collect information about:
\begin{itemize}
	\item page faults
	They allow us to see what leads to from of JVM resident memory.
	\item context switches
	Context switches are interesting to see code path leads to leaving the CPU.
	\item disk i/o requests
	Show code paths leading to IO operations such as blocking disk seek operation.
	\item TCP events
	Show code paths leading from high-level Java code to low-level system methods such as connect or accept, so we can reason about performance and good design of network communication in much more better detail.
	\item CPU cache misses
	Show code paths leading to cache misses. Using this information we can optimize the Java code to make better use of the existing cache hierarchy.
\end{itemize}

All the information bellow can be described on a special chart called Flame charts.
\subsubsection{Flame Charts}
Flame Chart is a concept by a developer Brendan Gregg. Flame graphs are aa visualization for samples stack traces, which allows the hot paths in the code to be identified quickly. The output of sampling of instrumentation profiler can be significantly big and therefore visualizing can help to reason about performance in more comfortable way. 

\begin{figure}
	\centering
	\includegraphics[scale=0.35]{flame_chart.png}
	\caption{Flame Graph example}
	\label{fig:flame_chart}
\end{figure}
The description:
\begin{itemize}
	\item Each box represents a function call in the stack
	\item The \textbf{y-axis} shows stack frame depth. The top function is the function which was at the moment of capturing this flame chart on the CPU. All functions underneath of it are its ancestors.
	\item The \textbf{x-axis} shows the population of traces. It doesn't represent passing of time. The function calls are usually sorted alphabetically.
	\item The width of each box represents the time the function was on CPU.
	\item The colors are not significant, they are just used to visually separate different function calls
\end{itemize}

Flame charts can be created in a few simple steps, but it depends on the type of profiler the user wants to use. 
\begin{enumerate}
	\item Capture stack traces
	For this step we can use profiler of our choice.
	\item Fold stacks
	We need to prepare the stacks so Flame graphs can be created out of them. For this, there are several scripts prepared for different profiler types.
	\item Generate the flame graph itself, again using the prepared script provided on the link above.
\end{enumerate}

Purpose of this really short section was just to introduce the idea of Flame charts since it's one of the future plans the thesis could be extended to support. For more information about the flame charts please visit the Brendan Gregg's blog.
\section{Bytecode Manipulation Libraries}
This thesis highly depends on the  for which the byte-code manipulation is a core feature. Since the work is written in Java, we are mainly interested in instrumentation and byte-code manipulation libraries based on Java. This section covers . The purpose of this section is to introduce 4 standard bytecode manipulation libraries - Javassist, ByteBuddy, CGlib and ASM - and give their comparison. Since it's a core feature of the whole platform and affect the performance and the usability of the whole platform, the library was thoroughly reviewed before selected. ByteBuddy was selected to be used in the thesis and the reasons why are mentioned bellow as well.

\subsection{ASM}
ASM is a low-level high-performance Java bytecode manipulation framework. It can be used to dynamically create new classes or redefined already existing classes. It works on the bytecode level so the user of this library is expected to understand the JVM bytecode in detail. ASM operates on event-driven model as it makes use of Visitor design pattern to walk through complex bytecode structures. ASM defines some default visitors such as \textit{FieldVisitor}, \textit{MethodVisitor} or \textit{ClassVisitor}. The ASM project can be a great fit for project requiring a full control over the bytecode creation or inspection since it's low-level nature.
\subsection{Javassist}
Javassist is well-known bytecode manipulation library built on top of ASM. It allows to Java programs to define new classes at runtime and also to modify a class files prior the JVM loads them. It works on higher level abstraction so the user of this library is not required to work with the low-level bytecode. The code to be injected to the existing bytecode is expressed as Java Strings which has the disadvantage that the code to be injected is not subject to code inspection in most of the current IDEs. The advantage of Javassist is that the injected code does not depend on the Javassist library at all. 
The strings representing the code are compiled on runtime by special javassist compiler which works well for most of the common programming structures but just to point out auto-boxing and generics are not supported by the compiler.
Javassist does not have support for the code injection itself. Therefore, it can be used for specifying the code which alters the original code but external tool needs to be used to inject the code.
\subsection{CGlib}
CGLib as another byte-code manupulation library built on top of ASM. The main concepts are build around `Enhancer` class which is used to create proxies by dynamically extending classes at runtime. The proxified class is then used to to intercept method calls and the result of previous methods or fields as we define. However cglib lacks comprehensive documentation making harder to even understand the basics from the users.

\subsection{Byte Buddy}
ByteBuddy is fairly new, light-weight and high-level bytecode manupulation library. The library depends only on visitor API of the ASM library which does not further have any other dependencies. It does not require from the user to understand format of java bytecode but despite this, it gives the users full flexibility to redefine the byte code according their specific needs. Also, classes created or instrumented by Byte Buddy does not depend on the Byte Buddy framework. Despite it's high-level approach, it still offers great performance and is used at frameworks such as Mockito or Hibernate. Byte Buddy can be used for both code generation and transformation of existing code.

\subsubsection{Code Generation}
Code generation is done by specifying from which class we want to create a subclass, in the most generic way we can create subclass from Object class. The newly created class can introduce new methods or intercept methods from it's super class. In order to intercept existing methods and change their behavior and return value, the method to be intercepted has to be identified using so-called ElementMatchers. These matchers allow us to identify methods using for example their names, number of arguments, return methods or associated annotations. The whole list of matchers and also examples how code can be generated is greatly described on the project Github page.

As mentioned earlier, the power behind Byte Buddy is also that it can be used to redefine classes at runtime. This is achieved several concepts, mainly Transformers, Interceptors and Advice API.
\subsubsection{Code Transformation}
In order to tell byte buddy when it needs to intercept a method or field, we need to identify the place in the code which triggers the interception. First,  the class containing method to be instrumented need to be located. It can be done by simply specifying the class name or using more complex structures. For example we can only consider all classes A extending class B whilst implementing interface C at the same type. 

The next step is to define the transformer class itself. Transformers are used to identify methods in the class which should be instrumented and they also specify what interceptor or advice should handle the instrumentation for the particular matched method. Such methods can be again identified using the ElementMatchers mentioned in the section above. In more detail, transformer interface has a method transfrom which has DynamicType.Builder as it's argument. This builder is used to create a single transformer wrapping all the previous ones for all classes in the code so the result of this builder can be thought of as a dispatcher of the instrumentation. Methods to be instrumented are specified on the builder instance using the ElementMatchers  as well as what interceptor or advice API will be used to handle the transformation.

As mentioned above there are 2 ways how to instrument a class in ByteBuddy.
\subsubsection{Interceptors}
Interceptor is a class defining the new or changed desired behavior for the method to be instrumented. 
We demonstrate how Byte Buddy uses interceptors on a small example. Let's assume we have original class Foo:
\begin{lstlisting}[language=Java]
class Foo {
	String bar() {
		return "bar"; 
	}
}
\end{lstlisting}
	
Le'ts also assume that our Interceptor is of type Qux. The interception of the class Foo using our interceptor looks like this in schematic code:

\begin{lstlisting}[language=Java]
class Foo {
	// Requires your interceptor class to be known
	static Qux $interceptor;
	String bar() {
		return $interceptor.intercept(); 
	}
	static {
		// Requires knowing the framework
		$interceptor = ByteBuddyFramework.defineField(Foo.class);
	}
}
\end{lstlisting}
		
We can therefore see that in case of interceptors, Byte Buddy does not inline the byte code to the \texttt{Foo} class but requires the interceptor class to be available on the machine where instrumentation takes place. Also the interceptor field needs to be initialized, which is in this case done in the static initializer. The initialization of interceptors is done using special helper class called \texttt{LoadedTypeInitializer}.

There are multiple ways how this behavior can be changed:
\begin{enumerate}
\item In Byte Buddy, initialization strategy can be modified according to the specific needs. We can define no-op strategy and read \texttt{LoadedTypeInitializer} right before the class is about to be instrumented.  Later we can perform the initialization our self using observed initializer or we can even serialize this initializer together with the \texttt{Qux} \texttt{Interceptor} class, send them to different JVM where the instrumentation should take place and manually initialize the the interceptor field.
\item Instead of referring to \texttt{Qux} as a instance, we can delegate to \texttt{Qux} as a class and call the interception logic via static methods. In this case the interceptor class still needs to ne known at runtime, but there is no need to perform the interceptor initialization.
\item Instead of using interceptors, advice api which inlines the code to the class itself may be used.
\end{enumerate}
\subsubsection{Advice API}
Advices are another approach how code can be instrumented in Byte Buddy. Compared to \texttt{Interceptor}s it is more limited, but on the other hand, in cases where it's possible to use it, the code is in-lined into the class's bytecode and therefore no other dependencies are required. It is also stated in Byte Buddy documentation that performance of Advice API is better compared to using interceptors.

However, instrumentation using Advice API is only allowed before or after the matched method which is achieved using the \texttt{Advice.onMethodEnter} and \texttt{Advice.onMethodExit} annotations.

\section{Communication Middleware}
This thesis consist of several parts which need be able to communicate. The communication is also complicated by the fact that the parts are written in different programming languages - Java and C++. In order to achieve communication in such an environment, following libraries has been inspected.
\subsection{Raw Sockets}
We are not referring to a library but to using raw sockets on their low-level API. Using raw sockets has several pros and cos. It give us a full flexibility and the highest possible performance since there isn't any additional layer between our application data and the socket itself. However, integrating different platforms and different languages can be time-consuming. Several frameworks have already been created to achieve this so the user does not need to know about the language or underlying platform.
\subsection{ZeroMQ}
ZeroMq is a communication library build on top of raw sockets. The core of the library is written in C++ however binding into different languages exist. The library is able transport messages inside a single process, between different processes on the same node, using TCP or also using multicast.

The library also supports to create typologies using one of many supported communication patterns like publisher-subscriber or request-reply.
\begin{itemize}
	\item Hiding the differences of underlying operating systems.
	\item Message framing - delivering whole messages instead of stream of bytes
	\item No need to worry about queuing messages. The internals take care of ensuring the messages are sent and received in correct order. The user can send the messages without knowing whether there are other messages in the queue or not.
	\item Language mappings to different languages.
	\item Ability to create a topologyy. For example, one socket can be connected to multiple endpoints.
	\item Automatic TCP re-connection
	\item Zero-copy
\end{itemize}
\subsubsection{Zero-copy in ZeroMQ}
The library also tries to apply concept called zero-copy if possible. When high-performance is expected from a system or network, copying of data is usually considered harmful and should be minimized as possible. The technique of avoiding copies of data is known as zero-copy.

Example of data copying is transferring data from memory to network interface or from user application to underlying kernel.  We can see that zero-copy can't be implemented at all layers because for example without copying the data from the kernel to network interface, we could not actually exchange any data. However, ZeroMQ can achieve zero-copy at least on the application message level so the users can create ZeroMQ messages from their data without any copying which is a big performance plus.
\subsection{NanoMsg}
\label{sec:nanomsg}
NanoMsg [http://nanomsg.org/documentation-zeromq.html] is a socket library shadowing the differences in the underlying operation systems. It offers several communication patterns, is implemented on C and does not have any other dependencies. Generally, it offers very similar features to ZeroMQ since it's heavily based on it.

Unlike ZeroMQ, nanomsg matches the full POSIX compliance. The author of the library states, that since it's implemented in C, the number of memory allocations is drastically reduced compared to c++ when using C++ STL containers for example. Also compared to ZeroMQ, objects are not tightly bound to particular threads this it gives the user flexibility to create their custom threading models without big limitations. NanoMsg should also implement zero-copy technique at additional layers which again leads to performance benefits.

As in ZeroMQ, NanoMsq supports the following transport mechanisms:
\begin{itemize}
	\item \textbf{INPROC}
	Used for transporting messages withing a single process, for example between different threads. In-process address is arbitrary case-sensitive string starting with \texttt{inproc://}
	\item \textbf{IPC} - inter processes communication
	It enables several processes to communicate on the same node
	The implementation uses native IPC mechanism available on the target platform. On Unix-like systems, IPC addresses are just references to files where both absolute and relative path can be used. The application has to have rights to read and write from the IPC file in order to allow the communication.
	On Windows, the named pipes are used. The address can be arbitrary-case sensitive string containing any character but backslash. On both mentioned platforms, the address has to start with \texttt{ipc://} prefix.
	\item \textbf{TCP}
	TCP is used to transport messages in a reliable manner to a single recipient to in a reachable network. When connecting to a node, the address in format \texttt{tcp://interface:port} needs to used and when binding a node, address in format texttt{tcp://*:port} should be used.
\end{itemize}

NanoMsg can be used in via it's core C library, but also several language mappings for different languages exist.
\subsubsection{C++11 Mapping}
Nanomsgxx [https://github.com/achille-roussel/nanomsgxx] is a C++11 mapping for nanomsg library. It is a small layer build on top of core library making the API more C++11 like friendly. Especially, there is no need to tell when to release resources, since it's handled automatically in desctuctors. The \texttt{nnxx::message} abstraction over NanoMsg \texttt{nn::message} automatically manages buffers for zero-copy and also errors are reported using the exceptions which are sub-classes from \texttt{std::system\_error}

\subsubsection{Java Mapping}
Several Java bindings of nanomsg library exists, but just jnanomsg library [http://niwinz.github.io/jnanomsg/latest/] is described here. This language binding is build on top of JNA - Java Native Access library. It offers all the functionalities offered by the core library but also introduces non-blocking sockets exposed via a callback interface.

\section{Java Libraries}
This section describes some fundamental Java related libraries on technologies on which this thesis heavily depends. Firstly, Java Virtual Machine Tool Interface (JVMTI) is described followed by basic introduction to Java Native Interface. Important Java concepts and classes relevant to the thesis are described in the following few sections.
\subsection{JVMTI}
The JVM Tool Interface https://docs.oracle.com/javase/7/docs/platform/jvmti/jvmti.html is a interface used by development and monitoring tools for communication with JVM. It allows its user to monitor and control the execution running in Java virtual machine. An application communicating with the JVM using JVMTI is usually called agent. Agents are notified of the events happening inside JVM and can react upon them.

Agents run in the same process as the application itself this reducing the communication. Since JVMTI as interface written in C, agents can be written in C or C++. The agent has to be attached to the application via 

JVMTI supports 2 modes how agent can can be started, either in OnLoad phase or in Live phase. In OnLoad phase, client is started together with the application and agent location can be specified using 2 arguments:
\begin{itemize}
	\item \texttt{-agentlib:<agent-lib-name>=<options>} \newline
	In this case, the library name to load is specified and it is loaded using platform specific manner .
	\item \texttt{-agentpath:<path-to-agent>=<options>} \newline
	In this case, the path to a location of the library is specified and the library is loaded from there.
\end{itemize}

In the Live phase, the agent is dynamically attached to running application. This approach can be though as more flexible since we don't have to specify agent library to monitored application in advance, but it brings several limitation as well.

We don't aim to describe full JVMTI functionality here, please consider this just as brief introduction to the interface and inspect the documentation for more information. In the following sections we aim to very briefly describe the important parts of JVMTI relevant to the thesis.

\subsubsection{JVMTI Agent Initialization}
When client is started, the method \newline \texttt{Agent\_OnLoad(JavaVM *jvm, char *options, void *reserved)} is called. In this method we can do custom initialization for our agent.

Usually the initialization consist of several phases:
\begin{enumerate}
	\item Optionally, parse arguments passed to JVMTI agent.
	\item Initialize JVMTI environment in order to be able to communicate with the observed application. JVMTI does not handle threads switches automatically, so proper locking and thread management fully depends on the user code.
	\item Register capabilities we want the JVMTI to support. We can specify what are the operations our JVMTI agent can perform. The agent can be for example allowed to re-transform classes, signal threads or generate different class hook events.
	\item Register events we are interested in observing. JVMTI does not inform the agent about all events by default, the events has to be manually defined.
	\item Register callbacks for the events we are interested in. In case of agent used for instrumentation we are mostly interested in events \texttt{cbClassLoad}, \texttt{cbClassPrepare}, \texttt{cbClassFileLoadHook}, \texttt{callbackVMInit} \newline
	and \texttt{callbackVMDeath}.
	\item Optionally, initializing phase is also good for creating various locks required for synchronization between different JVMTI threads.

\end{enumerate}

The user of JVMTI is also required to manually implement queening and locking when processing multiple JVMTI events at the same time since the framework does not handle this. http://www.oracle.com/technetwork/articles/java/jvmpitransition-138768.html
\subsubsection{JVMTI basic callbacks}
As mentioned above, there are several events send from the observed application. When instrumenting, we are mostly interested in the following events:
\begin{itemize}
	\item  \texttt{cbClassLoad} - triggered when class has been loaded by target JVM
	\item \texttt{cbClassPrepare} - triggered when class has been prepared by target JVM. At the point the class is prepared all static fields, methods and implemented interfaces are available but no code has been executed at this phase.
	\item \texttt{cbClassFileLoadHook} - triggered when virtual machine obtains class file data but before the class is loaded. Usually, class instrumentation happens based on this hook since the callback allows as to change the bytecode passed for further loading.
	\item  \texttt{callbackVMInit} - triggered when virtual machine is initialized
	\item  \texttt{callbackVMDeath} - triggered when virtual machine has been closed either using standard way or forcibly.
\end{itemize}

\subsection{JNI}
Java Native Interface is a framework which allows Java code running in a Java Virtual Machine to call native applications ( usually written in C or C++ ). It also allows native applications to access and call Java methods.

All operations require instance of class \texttt{JNIEnv}. This environment keeps the connection to the virtual machine. When calling the methods from native application, the method has to be first found. This is achieved by specifying  the types and method signature.
\subsubsection{Java Types Mapping}
For each Java primitive type there is corresponding native type in JNI. Native types always start with the \textbf{j} as the prefix, for example \texttt{boolean} is Java type whereas \texttt{jboolean} as native type.
All other JNI reference types are referred to via \texttt{jobject} class. This means that java arrays are accessed via \texttt{jobject} as well.

The most important is however how we can specify the types in method signatures. There is a mapping giving each type a signature which can be used exactly for this purpose. This table is base on http://docs.oracle.com/javase/7/docs/technotes/guides/jni/spec/types.html.
\begin{center}
\begin{tabular}{ l l }
	  \hline
	  Type Signature & Java Type \\ \hline
	Z & boolean \\
	B & byte \\
	C & char \\
	S & short \\
	I & int \\
	J & long \\
	F & float \\
	L fully-qualified-class ; & fully-qualified-class \\
	{[} type & type{[]}\ \\
	( arg-types ) ret-type & method type \\
\end{tabular}
\end{center}

So for example the method: \newline \texttt{xx.yy.Person foo(int n; boolean[] arr, String s);}
would have the following signature:

\texttt{(I[ZLjava/lang/String;])Lxx/yy/Person;}

Note that in JNI, the elements in fully qualified class name are separated by slashes instead of dots.
\subsubsection{Example JNI Method Call}
On the method bellow we can see how JNI can be used to call a Java method \texttt{getClassLoader}.

\begin{lstlisting}[language=c++]
        jobject getClassLoaderForClass(JNIEnv *jni, jclass clazz){
        // Get the class object's class descriptor
        // (jclass inherits from jobject)
        jclass clsClazz = jni->GetObjectClass(clazz);
        // Find the getClassLoader() method in the class object
        jmethodID methodId = jni->GetMethodID(	clsClazz,
																         "getClassLoader",
																         "()Ljava/lang/ClassLoader;");
        return (jobject) jni->CallObjectMethod(clazz, methodId);
        }
\end{lstlisting}

First we need to get a reference to a method, which we use later for the invocation itself. From performance reasons, it's good practice to cache the references to methods or objects in Java which we access from JNI often since getting the reference has some initial overhead.

\subsection{Relevant Aspects of the  Java Language}
This section covers selected areas of the Java programming language relevant to the thesis. It briefly describes the class loading process when for dynamically loaded classes. This is followed by explanation of 2 important class loaders relevant to the thesis and lastly, \texttt{ServiceLoader} class is shortly described.
\subsubsection{Class Loading Process}
Java allows program to load classes dynamically at runtime. This is achieved by a following process:
\begin{enumerate}
	\item \textbf{Loading} - Load the bytecode from class file
	\item \textbf{Linking} - Linking is the process of incorporating a new class to the runtime state of the JVM. It phase consists of 3 sub-phases:
	\begin{enumerate}
		\item \textbf{Verification} - Ensure that type in the binary former is correct and respects JVM restrictions.
		\item \textbf{Preparation} - This phase consist of allocation memory for fields inside the loaded type.
		\item \textbf{Resolution} - This phase is optional ( depends on JVM implementation ). Resolution is the process of transformation symbolic references in the type's constant pool into direct references. The implementation may decide to behave in lazy way and delay resolution for the time when the type is being actually used. Constant pool contains all references to variables and methods found during compilation time.
	\end{enumerate}
	\item \textbf{Linking Phase} - class variables are initialized to initial values
\end{enumerate}

Loading step loads data from class files in a binary from known as the bytecode.
\subsubsection{Relevant Class Loaders}
There are several class loaders used natively in Java. However we describe only 2 which are references in the thesis later. 

\begin{itemize}
	\item\textbf{ Bootstrap class loader} \newline
	This classloader is used to load system classes. When using native agent, even classes loaded by bootstrap classloader can be instrumented and thus behavior of standard Java classes can be changed.
	
	\item  \textbf{sun.reflect.DelegatingClassLoader} \newline
	This class loader is used on the Sun JVM as the effect of a mechanism called inflation. Usually reflective access to method or fields is initially performed via JNI calls. When Sun JVM determines there is a repetition in calling the same method or using the same field via JNI ( reflection), it creates synthetic class ( class created dynamically at runtime), which is used to perform this call without using JNI. This has initial speed overhead, but at the end it speeds up the reflection calls.
	The classes created for this purpose are loaded and managed by exactly this class loader. 
\end{itemize}
\subsubsection{ServiceLoader Class}
ServiceLoader class is used to locate and load service providers. Service Provider is an implementation of some service which is usually defined as set of methods. The service is often defined as abstract class or interface. 

ServiceLoader allows us to specify the service type for which we want to load all service providers and then load the desired service providers. The available services have to be defined in the META-INF folder of the application jar distribution. Let's say we have a service A and 2 implementations, Impl1 and Impl2. In that case META-INF folder with contain text file with name A containing lines
\begin{center}
 \texttt{Impl1} \newline
\texttt{Impl1}  \newline
\end{center}
Service loaders can therefore be used to extend the application without changing the source code. When the user of the application needs to provide another implementation of the service, it can create service provider, register it inside the META-INF folder and the application will use the new service provider as well the rest of the providers defined earlier.
\section{Logging libraries}
One of the key aspects of the developed platform is low-overhead. Logging can have negative effect on the performance but sometimes it's necessary to have information from various application runs. That is why the selection of logging library is important for the performance of the thesis as well. 

Spdlog is a C++11 fast, header only logging library this project is based on. It allows both synchronous and asynchronous logging and custom message formation
\section{Docker}
Docker is an open source project to pack, ship and run any application as a lightweight container [citate]. It is used to package the applications in a prepared environments so the user does not need to worry about configuration and downloading the correct dependencies for the application. 

Docker Compose is an extension build on top of docker allowing us to specify multi-container startup-script. This script can define dependencies between different containers which leads to a simple and automated way on how to start whole bunch of related applications in separated environments using one single call. 


\chapter{Analysis}
This chapter gives overview of two significant related platforms on which this work depends - Google Dapper and Zipkin. It continues with description of several background concepts and tools which are to some level relevant to the thesis, such as tools for large scale debugging, tools for visualizing the monitored data and also several profiling tools and their comparison. The libraries for bytecode instrumentation and different communication middle-wares are described in detail as the selected libraries affect the platform at very low level. This chapter ends with a comparison of different approaches to instrumenting Java applications.
\section{Related Work}
The most significant platforms to this thesis are Google Dapper and Zipkin, where Zipkin is based on the previous. Both serves the same core purpose which is to monitor large-scale Java based distributed applications. This thesis is based mainly on Google Dapper but also uses helpful Zipkin modules such as the user interface. Since Zipkin is developed according to Google Dapper design, these two platforms shares very similar concepts. The most important concept is a Span and it is explained in more details in the  \hyperref[subsec:Spans]{Spans} section. For now, we can think of a span as time slots encapsulating several calls from one node to another with well-defined start and end of the communication. The following two sections describes the basics of both of the mentioned platform.
\subsection{Google Dapper}
Google Dapper is proprietary software which was mainly developed as a tool for monitoring large distributed applications since debugging and reasoning about applications running on multiple host at the same time, sometimes written in different programming languages is inherently complex. Google Dapper has 3 main pillars on which is built:
\begin{itemize}
	\item Low overhead
	It was assumed that such a tool should share the same life-cycle as the monitored application itself thus low overhead was on of the main design goals as well. Google dapper 
	\item Application level transparency
	The developers and users of the application should not know about the monitoring tool and are not supposed to change the way how they interact with the system. It can be assumed from the paper that achieving application level transparency at Google was easier than it could be in more diverse environments since all the code produced in the Google shares the same libraries and control flow.
	\item Scalability
	Such a system should perform well on large scale data.
\end{itemize}	
Google Dapper collects so called distributed traces. The origin of the distributed trace is the communication/task initiator. This is demonstrated at the figure ...
	
There were two proposals for obtaining this information - using the black-box and annotation-based monitoring approaches. The first one assumes no additional knowledge about the application whereas the second can use of additional information via annotations. Dapper is mainly using black-box monitoring schema since most of the control flow and RPC subsystems are shared among Google.
	
	Trace trees and spans


\subsection{Zipkin}

\section{Background and Tools}
\subsection{Tools for Large-Scale Debugging}
\subsection{Tools for Visualizations the Captured Data}
\subsubsection{Flame Charts}
\subsubsection{Graphite and Graphana}
\subsection{Profiling Tools}
\subsubsection{System Profilers}
\subsubsection{JVM Profilers}
Write about AsyncGetCallTrace



\section{Instrumentation Libraries}
\subsection{Javassist}
\subsection{ByteBuddy}
\subsection{CGlib}
\subsection{ASM}
.. just give brief overview what were the instrumentation libreries choices. The selected one will be describied in the next section

\section{Communication Middleware}
give comparison between the possible communication middle-wares
\subsection{Raw Sockets}
\subsection{ZeroMQ}
\subsection{NanoMSG}

\section{Comparison of Agent Approaches}
give introduction to various instrumentation techniques and compare the 2 approaches
\subsection{Java Agent Solution}
\subsection{Native Agent Solution}


\chapter{Design}
\label{chap:design}
This chapter describes design of the whole platform in details, however implementation specifics of some parts of the Distrace tool are described in the following Chapter \ref{chap:implementation}. The current chapter starts with the high-level overview of the complete platform and interactions between its parts. It is followed by a simple use-case to give the reader an idea how the Distrace tool is can be used.

Spans and their format are described next, followed by design of the native agent and instrumentation server. This chapter ends by description of the default Zipkin user interface and also JSON format, in which the user interface accepts the data from the instrumentation server. 

\section{Overview}
\label{design:overview}
Main purpose of the Distrace tool is to collect distributed traces. In order to achieve that, the Distrace tool is based on the concept of spans. Spans are used to denote some specific part of the communication between the communicating nodes and are important elements for building the whole trace trees. Trace trees consists of several spans and represent the complete task or communication, where a span inside the trace usually represents a few remote procedure calls between two neighboring nodes. The node initiating the trace creates so called parent span and new calls started within the scope of this span create new nested spans. Created spans can be exported using different span exporters and can be send to the user interface using various data collectors. Span exporters are used to export spans in desired format on disk or the network for further data collection. The collected data are used for spans visualization in the user interface. The user interface receives the spans from the span exporters or data collectors and present them to the user in a form of trace trees.

Definition of when a new span is to be created and when an existing span needs to be closed is done by a developer by extending the core instrumentation server library. The created extended instrumentation server is then used for instrumenting the classes of the original application, however spans are still located in the scope of the application itself. In order to obtain the class files for transformation, the native agent runs as part of the monitored application and sends the desired classes to the instrumentation server. The native agent is a core part of the whole platform. It is attached to the monitored application and additionally to providing the data to the instrumentation server, it is used to obtain various low-level information from the application. 

The Distrace tool therefore consists of three main components:
\begin{itemize}
	\item Native Agent
	\begin{itemize}
		\item Is used to obtain byte-code for the instrumentation.
		\item Is used to actually apply the instrumented byte-code.
	\end{itemize}
	\item Instrumentation Server
	\begin{itemize}
		\item Instruments the classes obtained from the native agent.
		\item Is also base library for custom application instrumentation server.
		\item Can contain implementation of customized span exporters.
	\end{itemize}
	\item User Interface
\end{itemize}


The Figure \ref{fig:architecture} denotes the basic relationships between the major parts. Instrumentation server communicates with the native again, mainly in order to instrument classes. The application communicates with the user interface by sending the spans to it. The spans can be send either via data collection agent or via one of the default span exporters explained later in this chapter. Each part is described in more detail later in this chapter. \begin{figure}
	\centering
	\includegraphics[scale=0.6]{architecture.png}
	\caption{Basic relationship between the major components. }
	\label{fig:architecture}
\end{figure}

The Figure \ref{fig:full_overview} shows how trace trees and spans are related on a simple two nodes example. Each trace is separated from each other and represents tracing of a single computation, which consist of several spans. Spans denote more local computation and can also contain additional application-specific information. In order to connect the information from multiple nodes, the trace information needs to be attached to the node communication. That is also a reason why in this case the methods \texttt{send()}, \texttt{receive()} and \texttt{process()} need to be instrumented. These methods also open and close spans at the correct places in the code.
\begin{figure}
	\centering
	\includegraphics[scale=0.56]{full_overview.png}
	\caption{Trace and span demonstration. }
	\label{fig:full_overview}
\end{figure}
\section{Example Use Case}
\label{design:use_case}
In order to demonstrate an use case for which this architecture may a good fit, a small example on a simple distributed application is shown in this section. The example consists of three modules. The client module used for submitting tasks, the execution module and the module used for exporting the data. These modules can be represented as separated threads in a single application or as different nodes of the distributed application. In this example, the user always passes a task to the client module. This module performs some pre-processing and sends the task to the execution module. This module performs the computation and once it's done, sends the data to the exporter module, which exports the data on disk and informs the client of the task completion. The architecture of the example can be seen on the Figure \ref{fig:example_arch}.

The goal of this example is to record and visualize how long the transfers between different modules last and how long the processing on each module takes. It is also assumed that the platform does not collect this information already, otherwise the cluster monitoring tool would not be required. For simplicity, let's also assume that each module performs the functionality in a single method. The following code sections give the schematic code of each method.

	\begin{figure}
		\centering
		\includegraphics[scale=0.5]{example_arch.png}
		\caption{Example architecture.}
		\label{fig:example_arch}
	\end{figure}



\begin{itemize}
\item \textbf{The client:}
\begin{lstlisting}[language=Java]
public acceptTask(Task task){
  preprocessTask(task)
  ...
  sendTaskForComputation(task)
  ...
  waitForExporterToFinish()
  ...
}
\end{lstlisting}

\item \textbf{The executor:}
\begin{lstlisting}[language=Java]
public execute(Task task){
  TaskResult result = executeTask(task)
  ...
  sendResultForVisuzalization(result)
}
\end{lstlisting}

\item \textbf{The exporter:}
\begin{lstlisting}[language=Java]
public export(TaskResult result){
  saveToDatabase(result)
  ...
  notifyClient()
}
\end{lstlisting}
\end{itemize}

In order to collect this type of information and be able to reason about the relationship between the modules, these methods needs to be instrumented. The instrumented code should look as in the schematic code below. Generally, the logic which keeps the track of the current trace and span needs to be injected into the code. To achieve this, developers are supposed to extend the core instrumentation server, which acts as the base library and provides them with several helper methods used to specify the instrumentation points for their applications. 

\begin{itemize}
\item \textbf{The instrumented client:}
\begin{lstlisting}[language=Java]
public acceptTask(Task task){
  TraceContext tc = TraceContext.create()
  tc.attachOnObject(task)
  Span s = tc.openSpan("Main Client Span")	
  s.addAnnotation("tskReceived", timestamp)
  ...
  preprocessTask(task)
  s.addAnnotation("tskPreprocessed", timestamp)
  ...
  sendTaskForComputation(task)
  ...
  Result res = waitForExporterToFinish() // blocking method
  ...
  tc.closeCurrentSpan()
}
\end{lstlisting}
The \texttt{create} method creates a new trace and the \texttt{attachOnObject} method attaches the trace context on the \texttt{task} object, which is passed around the network. The method \texttt{openSpan} opens a new span encapsulating the client computation. The \texttt{addAnnotation} method is used to add application specific information to the current span. The \texttt{closeCurrentSpan} method is used to close the current span and export the content using the provided span exporter. In the default case, the data are sent to the user interface directly.


\item \textbf{The instrumented executor:}
\begin{lstlisting}[language=Java]
public execute(Task task){
  TraceContext tc = TraceContext.getFromObject(task)
  tc.openSpan("Executor span")
  TaskResult result = executeTask(task)
  tc.attachOnObject(result)
  ...
  sendResultForExport(result) // non-blocking method
  tc.closeCurrentSpan()
}
\end{lstlisting}
In this case, we don't create a new trace context, but obtain the existing one from the \texttt{task} object on the input. A new nested span is opened within the scope of the span, created in the previous module. The current span marks the span from the previous module as its parent.

\item \textbf{The instrumented exporter:}
\begin{lstlisting}[language=Java]
public export(TaskResult result){
  TraceContext tc = TraceContext.getFromObject(result)
  tc.openSpan("Exporter span")
  saveToDatabase(result)
  ...
  tc.closeCurrentSpan()
  notifyClient()	
}
\end{lstlisting}
In this case, the meaning of the methods is the same as above.

\end{itemize}

The developer should extend the base instrumentation server to instrument the classes in order to have a similar format as above. The extended instrumentation server is described on the Figure \ref{fig:example_extended}.

	\begin{figure}
		\centering
		\includegraphics[scale=0.5]{example_extended.png}
		\caption{Structure of the extended instrumentation server JAR artifact.}
		\label{fig:example_extended}
	\end{figure}

The extended instrumentation server is run on each node or on the network and is used to perform the instrumentation requested from the application. The native agent has to be attached to all nodes of the distributed application prior its start and the path to the extended instrumentation server JAR needs to be set as the mandatory argument. The default span exporter is used in this case and the collected spans are sent right to the Zipkin UI. The default IP address and port of the user interface is used when it's not explicitly configured as the native agent argument.

A single collected trace from this application should look as shown on the Figure \ref{fig:example_trace}.

	\begin{figure}
		\centering
		\includegraphics[scale=0.5]{example_trace.png}
		\caption{Example trace in case of the example application.}
		\label{fig:example_trace}
	\end{figure}

Therefore, it can be seen that the only part the developer needs to work on is the extension of the instrumentation server to specify the custom instrumentation points, otherwise the rest of technical work is done automatically. The end user is only responsible for starting the application with the agent attached.

\section{Spans and Trace Trees}
\label{subsec:spans}
As mentioned briefly in the previous section, spans are used to gather the information about the distributed calls or so called, distributed stack traces. Points in the code where spans are created and closed are defined as part of the instrumentation server but since it's the most important concept in the thesis, we explain them in the separated section. 

Spans are the main concept behind capturing the distributed traces. They are entities injected to the code of the instrumented application to keep track of the communication and state between the nodes of the distributed application. Usually, the initiator creates so called parent span and new calls started within the span create new nested spans. Collected spans can be processed using different span exporters and can be sent to the user interface using various data collectors.

Span has several mandatory and optional attributes. The mandatory attributes are trace id, span id and parent span id. Trace id identifies one complete distributed call among all interacting nodes of the cluster. This attribute is attached automatically when a new root span is created. A root span is the first span created inside the trace and does not have parent id attribute set up. Therefore, the user interface back-end can distinguish between regular spans and root spans and can identify the start of the whole trace. Parent id of a span is id of span in which scope the child span was created. The span and its parent span can be located on the same node or on different nodes as well. The first variant can be useful in cases where the developer requires to trace multiple threads as separated spans within a single application node.  

Span has also several additional optional fields, which are later used in the user interface. The fields are:
\begin{itemize}
	\item \textbf{Timestamp} - when the span started.
	\item \textbf{Duration} - how long the span lasted.
	\item \textbf{Annotations} - annotations used to carry additional timing information about spans. For example, time when the span has been received on the receiver side or the time the span has been processed at the receiver side can be set using these annotations.
	\item \textbf{Binary annotations} - annotations used to carry around application specific details. They can also be used to transfer information between communication nodes inside of a single span. For example, a sender can store number of bytes sent during the request and a receiver can use this information to calculate overall number of bytes received from this particular node.
\end{itemize}
Each span has also an internal field called \textbf{flags}. The developer may store tags important to the instrumentation and these flags are transferred as part of the spans. Flags are not sent to the user interface and are only used for instrumentation purposes. For example, they can be used in case of multiple spans exist at a single moment. In this case, these spans can be annotated with special flags and the decision can be made based on this to process these spans.

Each annotation, both binary and regular, has also an endpoint information attached. This element consist of:
\begin{itemize}
	\item \textbf{IP} - IP of the node on which this event was recorded.
	\item \textbf{port} - port of the service which recorded the span.
	\item \textbf{service name} - a special name, which is used in the user interface to group and filter different traces by names.
\end{itemize}

\subsection{Span IDs}
It is also important to mention that span id is created randomly. This is done in order to allow parallel spans to coexist in the same control flow without overlapping as can be seen on the Figure \ref{fig:parallel_spans}.

	\begin{figure}
		\centering
		\includegraphics[scale=0.6]{parallel_spans.png}
		\caption{Generating span ids randomly ensures that they don't overlap when they are created in parallel.}
		\label{fig:parallel_spans}
	\end{figure}
If the ids were not random at different nodes of the distributed system, the parallel spans would be creating child spans with ids in the same linear sequence and therefore these spans would be overlapping, as can be seen on the Figure \ref{fig:parallel_spans_overlapping}
	\begin{figure}
		\centering
		\includegraphics[scale=0.6]{parallel_spans_overlapping.png}
		\caption{Generating span with the same linear sequence leads to span overlapping.}
		\label{fig:parallel_spans_overlapping}
	\end{figure}
	
The following sections contain information about how spans are exported for processing outside of the Distrace tool and also how spans are created using \texttt{TraceContext} and \texttt{TraceContextManager}.
\subsection{Span Exporters \& Data Collectors}
Spans can be exported from the application using so called span exporters. The type of the exporter can be configured via one of the native agent configuration properties. It is important to mention that the exporter is configured globally and each span in the application is using the same exporter\footnote{Exporter is a static attributed of the \texttt{Span} entity.}. The monitored application needs to have the chosen span exporter available at run-time, since the export is performed at the scope of the monitored application. Therefore, exporters are sent to the native agent during the initialization phase and the agent puts the exporters on the application classpath.

The Distrace tool includes two default implementation of span exporters, but also allows the developer to create new span exporters. Custom span exporters may be useful in cases when the developer wants to export the span data in a format used by different user interface or to use custom data collector. Data collector is a service, which collects data from the specified location and stores them in a central data storage available to all nodes in the distributed application. The Distrace tool does not implement data collection service as many services exist for this purpose. 

Data collected within spans are internally represented in JSON format understandable to the Zipkin user interface. This is also the reason why Distrace contains support for working with JSON data and it is explained in more detail later in the Section \ref{json_gen}. This output format can be customized by the custom span exporter.

The implementation details are available at the Section \ref{imp:exporter}.
\subsection{Trace Context}
Trace context is used for storing the information about the current span and also for creating new spans and closing current spans. Trace context can be attached to a specific object and thread. This is done in order to allow multiple threads to have different computation state and therefore the platform is able to capture multiple distributed traces at the same time on the same node. Trace context manager is used for attaching trace context to threads and vice-versa. It provides several operations allowing to the developer to attach trace context to a specific thread and also to get trace context from a specific thread.

Each trace is represented by trace context and is uniquely identified by Universally unique identifier (UUID) of type one\footnote{More UUID versions exist and are created based on different information}. This UUID type combines 48-bit MAC address of the current device with the actual timestamp. This way it is ensured that two traces created at the same time on different nodes can not have the same identifier. The identifiers are generated in the native agent using C++ library called Sole\footnote{The Sole library is available at \url{https://github.com/r-lyeh/sole} and can be used to create identifiers in C++ language.} and are made available to the Java code via published native operation \texttt{getTypeOneUUID}.

The trace contact has \texttt{openNestedSpan} and \texttt{closeCurrentSpan} operations. The first operation is used to create a new nested span and set the newly created span as the current one. Nested span is opened within the scope of its parent span and remembers the parent span. A root span is created in case no current span exists. The second operation is used to close the current span.

The second operation closes the the current span. This consist of two actions. The current span is exported using the configured span exporter and the parent span becomes the current span. The Figure \ref{fig:closing_spans} shows how spans are created, closed and exported. \begin{figure}
	\centering
	\includegraphics[scale=0.6]{closing_spans.png}
	\caption{Creating and closing spans.}
	\label{fig:closing_spans}
\end{figure}

\subsection{Transferring Span Information}
In order to capture the shared state between the nodes in the distributed application or between the threads on the same application node, the span details and trace context have to be transferred between the threads or the nodes. Distrace prepares several operations for attaching trace context to either a current thread or to an object acting as the carrier of the trace information.

Usually when transferring the trace context between the threads on the same node, the copies of the trace context should be used. This is not an issue when transferring the trace context between the different nodes of the distributed application.


The developer responsible for extending the instrumentation server can use operations like:
\begin{itemize}
	\item  \texttt{getTraceFromThread}
	\item \texttt{getTraceFromObject}
	\item \texttt{attachTraceToThread}
	\item \texttt{attachTraceToObject}
\end{itemize}
More information about the trace context API is available at \ref{imp:trace_context_api}.

There are also different variants of how spans can be closed. A span can be closed by the same node or thread who created it. In this case, attaching the trace context information to the current thread is usually sufficient. However, it is also possible that the span can be closed by different thread or node that originally created this span. In this case, the copy of the trace context should usually be created and attached to the object, which is used as the carrier of the trace context.
\section{Native Agent}
\label{native_agent_design}
The native agent is used for accessing the internal state of the monitored application and also to instrument classes so they can carry the span and trace identifiers between the application nodes. The main task of the agent is to check whether a class is required to be instrumented and if yes, send the class for the instrumentation to the server and wait for the instrumented code.

The native agent consist of several modules. The most important ones are:
\begin{itemize}
	\item \textbf{Bytecode parsing module}. \newline The classes in this module are used to parse the JVM bytecode in order to discover the classes dependencies for further instrumentation. Byte code parsing is a technical task described in the Section \ref{imp:parsing}.
	\item \textbf{InstrumentorAPI}. \newline The \texttt{InstrumentorAPI} class provides several methods, which are used to communicate with the instrumentation server JVM. All the queries to the server go through the instance of this class.
	\item \textbf{AgentCallbacks}. \newline All callbacks used in the native agent are defined in this namespace.
	\item \textbf{AgentArgs}.  \newline The \texttt{AgentArgs} class contains all the logic required for argument parsing.
	\item \textbf{NativeMethodsHelper}. \newline The \texttt{NativeMethodsHelper} class is used for registering native methods defined in C++. These methods can be used from the Java code without worrying about the low-level implementation.
	\item \textbf{Utilities module}. \newline This module contains several utility namespaces. The most important are \texttt{AgentUtils} and \texttt{JavaUtils} namespaces. The first contains methods for managing the JVMTI connection and for registering the JVMTI callbacks and events. The second is used to simplify work with Java objects in the native code via JNI.
\end{itemize}

\subsection{Agent Initialization}
The agent is initialized through the same phases as described in the Section \ref{subsec:jvmti_init}. The following JVMTI events are especially important to the thesis: \texttt{VM Init}, \texttt{VM Start}, \texttt{VM Death}, \texttt{Class File load Hook}, \texttt{Class Prepare} and \texttt{Class Load}. Callbacks are registered for all the mentioned events so the native agent can react to them accordingly in the code.

As part of the initialization process, the agent is responsible for either connecting to or starting a new instrumentation server. In case the native agent was started in the shared mode of the instrumentation server, the agent tries to connect to already existing server and the server is shared between all  nodes of the distributed application. In the local instrumentation mode, the server is started as a separated process automatically and the connection is established with the server using the inter process communication. In this case, each application node has dedicated instrumentation server.

The callback registered for the \texttt{VM Init} event is responsible for loading all additional classes from the instrumentation server as part of the initialization as well. The additional classes are for example \texttt{Span}, \texttt{TraceContext} or custom implementations of \texttt{SpanExporter} abstract class. These classes need to be available at run-time at the monitored application. They are required since the instrumented code is using trace context and spans for preserving the information about the current trace and the exporters for exporting the spans from the application. Therefore, these classes have to be available at the monitored application. The native agent is designed in a way that developers are not supposed to change the code of if. All the extension are supposed to be done as part of the extended instrumentation server. Therefore, the server is asked at the initialization phase for the list of all additional classes and they are sent to the native agent. The agent puts all the received classes on the application's classpath so they are available to the instrumented code.

\subsection{Instrumentation}
Code for handling the instrumentation is part of the callback for the \texttt{Class File load Hook} event. The callback has the bytecode for the class being loaded as its input parameter and allows the developer to pass a new instrumented bytecode as the output parameter. The process of instrumentation is described at this section, however some technical details are described in the Chapter \ref{chap:implementation}.

The process consist of several stages:
\begin{enumerate}
	\item Enter the critical section. It can happen that the class file load hook is triggered multiple times and to prevent confusing the instrumentation server, the lock has to be acquired before the instrumentation of a class starts.
	\item Firstly, we check whether the virtual machine with the application is started. If the JVM is initialized, the instrumentation continues, otherwise the instrumentation for currently a class currently being loaded is skipped. In this case, the instrumentation server is not contacted since the classes being loaded are at this point system classed and it is not desired to instrument Java system classes.
	\item Attach JNI environment to the current thread. Since the JVMTI and JNI does not have automatic thread management, it is up to the developer to take care of correct threading management.
	\item Discover the class loader which is loading the class.
	\item Parse name of the class currently being loaded. Even though the callback provides input parameter, which should contain the name of the class, at some circumstances it can be set to \texttt{NULL} even though the class name is available in the bytecode. Instead of relying on this parameter, the bytecode is parsed and the class name is found manually.
	\item Decide whether the instrumentation should continue. This check is based on the used class loader and name of the class being loaded. Classes loaded by the \texttt{Bootstrap} class loader and in case of Oracle JVM, classes loaded by \texttt{sun.reflect.DelegatingClassloader} are not supposed to be instrumented. 
	The \texttt{Bootstrap} class loader is used to load system classes and the second mentioned class loader is used to load synthetic classes. and in both cases, it's not desired to instrument classes loaded by these class loaders.
	There are also some ignored classes for which the instrumentation is not desired. For example, classes loaded during initialization phase from the instrumentation server and the auxiliary classes generated by the Byte Buddy framework should not be instrumented. Auxiliary classes are small helper classes Byte Buddy is using for example to access the super class of the currently being instrumented class. Therefore, the instrumentation continues only if the class is not ignored and not loaded by any of the ignored class loaders.
	\item The instrumentation server is asked whether it already contains the loaded class and also if the class should be instrumented. The agent does not know which classes are to be instrumented and therefore, it needs to query the server. The classes to be instrumented are marked by developer when extending the instrumentation server library using simple Byte Buddy API. 
	
	If the server does not contain the class, the native agent sends the class data to the instrumentation server, parse the class file for all the dependent classes and send all dependent classes for the instrumentation. This step is repeated throughout the dependency scan recurrently until the loaded class does not have any other dependencies or until all dependencies are already available on the server. All dependencies for the currently instrumented class have to be available on the server in order to perform the instrumentation.

	\item At this stage, the class is already on the instrumentation server and all dependencies for this class as well. The native agent waits for the instrumented bytecode to be sent from the server. 
	\item Exit the critical section.
\end{enumerate}

Several technical difficulties had to be dealt with during the development. For example, cyclic dependencies between classes need to be properly handled during the instrumentation. We need to also ensure that the dependencies for instrumented classed are also instrumented in the correct order. The final solution to these problems is described in the Section \ref{imp:native:inst}

The behavior of the agent may be configured using the arguments passed to the agent. Please see the Attachment 4 for the full list of native agent arguments.

\section{Instrumentation Server}
\label{sec:inst_server}
The instrumentation server is responsible for instrumenting the bytecode received from the native agent in separated JVM and it also acts as the base library for instrumenting specific applications. The developer extending the instrumentation server can use prepared operations to define custom instrumentation points without touching the internals of the native agent.

This section covers several design aspects of the instrumentation server, leaving the implementation details on the following sections. The core instrumentation on the server is handled by the Byte Buddy code manipulation framework. The native agent asks the server if the class currently being loaded is required to be instrumented. If yes, the server receives the bytecode, performs the instrumentation and sends the data back to the agent. The server does not contain any application state, in particular it does not take track about the distributed traces. The information about traces is contained in the instrumented classes within the monitored application.

The platform was designed to be configurable and deployment of the instrumentation server is supported via two approaches. The instrumentation server can be located either on the network to all nodes of the distributed application and can be shared by all nodes. This has the advantage of caching the instrumented classes. When any class is instrumented for the first time based on request from any node, it is saved and the instrumentation is not performed for other nodes. Instead, the class can be sent immediately. The disadvantage of this solution is higher latency between the agent and the instrumentation server since they are usually not on the same physical node. In this case, the instrumentation server has to be manually started in advance. Architecture of this scenario is depicted on the Figure \ref{fig:shared_server}.
 
 \begin{figure}
 	\centering
 	\includegraphics[scale=0.8]{shared_server.png}
 	\caption{Architecture with the shared instrumentation server. The dotted lines represent the communication between the server and the agent, whilst the regular lines represent data transfer from the agent to the user interface.}
 	\label{fig:shared_server}
 \end{figure}
 
 The other deployment method is that the instrumentation server runs on each node of the distributed application. This has the advantage of faster communication since in this case, inter-process communication is used to communicate between monitored JVM and the instrumentation server. The disadvantage of this solution is that all classes have to be instrumented on each node since there is no communication between the instrumentation servers. In this solution, the server is started automatically during the native agent initialization. Architecture of this scenario is depicted on the Figure \ref{fig:separated_server}.
 
 \begin{figure}
 	\centering
 	\includegraphics[scale=0.8]{separated_server.png}
 	\caption{Architecture with separated instrumentation server. The dotted lines represent the communication between the server and the agent, whilst the regular lines represent data transfer from the agent to the user interface.}
 	\label{fig:separated_server}
 \end{figure}

Except from the cached classes, the server does not contain any application state and it just reacts to the agent requests. It can accept four type of requests:
\begin{itemize}
	\item Request for code instrumentation.
	\item Request to store bytecode for a class on the server.
	\item Request to send all helper classes needed by the agent such as the \texttt{Span} class or \texttt{TraceContext} class.
	\item Request to check whether the server contains specific class or not.
\end{itemize}
The server interacts in more ways with the agent, however all communication is initiated by one of these four request types.	

The instrumentation server needs to deal with several technical problems. The main issue is that the classes, which are about to be instrumented, require all other dependent classes to be available. The other issue is instrumenting the classes with circular dependencies. The server also performs several optimizations to provide faster response to the agent such as caching the instrumented classes and minimizing the communication when possible. The technical aspects of these issues and the optimizations mentioned above are described in the Chapter \ref{chap:implementation}.
\subsection{Instrumentation}
The instrumentation of the class is triggered by the agent and it's done in two stages. The first stage informs the agent whether the class is already available on the instrumentation server or not. The second stage is the instrumentation step itself. The first stage is initiated by the agent and the server performs the check for class availability in three phases:
\begin{enumerate}
	\item Check if the instrumented bytecode for this class is available.
	\item If not, check if the original bytecode for this class is available.
	\item If not, check if the class can be loaded using the server's context class loader. This handles the cases where the user builds the instrumentation server together with the application classes or adds the classes on the server classpath for optimization reasons.
\end{enumerate}

The server informs the agent if it does not have the bytecode for the class available and in that case  the agent sends the class to the server. The server registers the received bytecode under the class name. The agent therefore does not have to send the class next time since it's already cached on the instrumentation server.
The second stage follows the first stage immediately. If the server already contains the instrumented class in the cache, the instrumented class is sent right away without instrumenting the class again. If the cache is empty, the class is instrumented and put into the cache.

The code to be instrumented is specified using the \texttt{MainAgentBuilder} and \texttt{BaseAgentBuilder} classes.
The instrumentation server expects the instance of \texttt{MainAgentBuilder} on the input of its \texttt{start} method. This is an abstract class containing single abstract method \texttt{createAgent(BaseAgentBuilder builder, String pathToHelperClasses)}, where the builder is a wrapper around the Byte Buddy \texttt{AgentBuilder} class, which is used to define the class transformers.

The developer needs to implement this method and specify on which classes and on which methods the instrumentation should happen. Since Byte Buddy is used for writing transformers and interceptors, please read more about Byte Buddy in the Section \ref{sec:byte_buddy}. The server provides several helper methods for creating the transformers and interceptors, which are less verbose then the standard Byte Buddy approaches.

Each transformer has to have associated either an interceptor or advice defining the code to be injected to the original code. Each interceptor implementation has to implement \texttt{Interceptor} interface. This is required for the server to be able to discover all interceptors at run-time without the need for changing the internals of the server. Each implementation of the interceptor needs to register itself in the META-INF directory of the generated JAR in the same way as the span exporters mentioned in the previous section. Custom service loader is then used to locate all classes implementing the \texttt{Interceptor} interface. 

The advices may be used without any special annotations since Byte Buddy in-lines the code defined by the advices into the original code and therefore there is no need to transfer the Advice classes to monitored application.

Even though Byte Buddy takes care about the internals of the instrumentation, the \texttt{BaseAgentBuilder} class is internally properly configured so the instrumentation is defined exactly as desired. The class implements four Byte Buddy listeners used for informing us about the instrumentation progress and allow us to react on the process of the instrumentation. The listeners are:
\begin{itemize}
	\item \texttt{onTransformation} listener is called immediately before the class is instrumented.  Implementation of the listener in the thesis also sends the agent all auxiliary classes required by the instrumented class and the initializers used for setting the static interceptor field on the instrumented class.
	\item \texttt{onIgnored} listener is called when the class is not instrumented. The class is not instrumented when the user does not define any transformer for the specified class.
	\item \texttt{onError} listener is called when some exception occurred during the instrumentation.
	\item \texttt{onComplete} listener is called when instrumentation process completed. It is called after both of \texttt{onTransformation} and \texttt{onIgnored} listeners.
\end{itemize}

Byte buddy requires dependencies for the instrumented class to be available. They are needed because the instrumentation framework needs to know signature of all methods in several cases, for example when the method is overridden in the child class. The dependencies are all the classes specified in the class file such as type of the methods return value or arguments, super class or implemented interfaces. 
By default, Byte Buddy tries to find these dependencies using two classes - \texttt{LocationStrategy} and \texttt{PoolStrategy}. The first class is used to tell Byte Buddy where to look for the raw bytecode of dependent classes. The classes are loaded by context class loader by default, but since the classes are received over the network, custom \texttt{InstrumentorClassloader} class loader is used to handle the class loading. It is a simple class loader which keeps the cache of the classes received from the agent and when a request for instrumentation comes, instead of looking into the class files, it loads the data from the cache in the memory.

However, Byte Buddy internal API does not work with raw bytecode for scanning the further dependencies and obtaining the metadata for the classes. It uses classes \texttt{TypeDescription} and \texttt{PoolStrategy} for this purpose. The first class has a constructor accepting the \texttt{Class} class and created instance contains metadata for the class such as the signature of all methods and fields, list of all interfaces or for example list of constructors. The second class is used for caching the type descriptions so they are not created every time the class is accessed. 

So in overall, class lookup is done in the following two steps:
\begin{enumerate}
	\item Check whether type description for the class is available. If yes, load the type description from the cache.
	\item If the type description is not available, load the class using the \newline \texttt{InstrumentorClassloader}, create type description for the class and put it in the cache.
\end{enumerate}

\subsection{Custom Service Loader}
In order to allow the developer to extend the base instrumentation library service loaders for loading the extensions are used. The service loader is used for two object types: 
\begin{itemize}
	\item \textbf{Custom span exporters} - Each span exporter inherits from the abstract class \texttt{SpanExporter}.
	\item \textbf{Custom Interceptors} - Each interceptor has to implement the interface \texttt{Interceptor}.
\end{itemize} 
The user can create custom span exporter and interceptors by either inheriting the desired class or implementing the required interface and put the name of the class inside the text file in the META-INF directory in the JAR file. The text file has to have the same name as the abstract class or the interface the implementation is for. For example, when user creates a new Interceptor called \texttt{x.y.InterceptorA}, the file \texttt{Interceptor} in the META-INF folder has to contain line \textbf{x.y.InterceptorA}.

Java provides service loader for this purpose. However the standard Java implementation looks up the classes defined as above and automatically creates new instances using the well-known constructors. For the thesis purposes this was unwanted as it is only required to obtain \texttt{Class} object representing the available implementation. Therefore a custom service loader was created for this purpose. This loader works in very similar way as the standard Java one, but instead of returning the instances of loaded services it just returns classes of available services. 

\subsection{JSON Generation}
\label{json_gen}
The data inside spans are internally stored as instances of \texttt{JSONValue} class since in order to support the communication with the default Zipkin UI they need to be exported as JSON. JSON is a lightweight format for exchanging data where the syntax is based on Javascript object notation.

The JSON handling is based on the minimal-json library\footnote{The library is available at \url{https://github.com/ralfstx/minimal-json}.}, however custom simplified implementation was created which fits the theses requirements. Also the number of dependencies is lowered by this decision. 

This JSON support is designed via several classes:
\begin{enumerate}
	\item \textbf{JSONValue}. The abstract ancestor of all JSON types. This type defines common methods to all implementation.
	\item \textbf{JSONString}. Class representing the string type.
	\item \textbf{JSONNumber}. Class representing the numeric types.
	\item \textbf{JSONLiteral}. Class representing the literals \textbf{null}, \textbf{true} and \textbf{false}.
	\item \textbf{JSONArray}. Class representing the JSON arrays. It has support for adding new elements into the array.
	\item \textbf{JSONObject}. Class representing the JSON objects. It has support for adding a new items in the object.
\end{enumerate}

Each \textbf{JSONValue} can be printed as valid JSON string where the printing is driven by\texttt{JSONStringBuilder} class. This class is also responsible for escaping the characters according to JSON standards. The default printer prints the data without any formatting as one line, however \texttt{JSONPrettyStringBuilder} prints the data in more human-readable format. The second printer is usually used for debugging purposes and the first one for real usage as the size of the data is smaller in this case.

\section{User Interface}
\label{sec:zipkin_ui}
The user interface receives spans and presents them in a hierarchical way so the relationships between different nodes can be seen easily. The important feature of the user interface is that the data for a single span can be sent incrementally. This means that several JSONs representing the same span can be sent with different annotations and the user interface merges these spans into single one and presents all annotations under the given span. This allow the Distrace tool to send part of data from the sender side and part of data from the receiver side directly to the user interface instead of sending the data back on forth to send them as one single complete span.

The Distrace tool is using Zipkin as default user interface. The default data format for exporting spans is designed in order to be understandable by this user interface. The user is however still able to change the data format to support custom user interface via custom span exporter class. This section gives an overview of Zipkin user interface and describes the Zipkin data model


\begin{figure}
	\centering
	\includegraphics[scale=0.5]{zipkin_ui_example.png}
	\caption{Example of Zipkin UI}
	\label{fig:zipkin_ui}
\end{figure}

Each span in the UI is clickable and all the additional information can bee seen at that level. In this thesis the stack trace are also collected at each span for monitoring purposes. Example of such information screen can be seen on the Figure \ref{fig:zipkin_ui_detail}.
\begin{figure}
	\centering
	\includegraphics[scale=0.4]{zipkin_ui_detail.png}
	\caption{Example of the detail span information.}
	\label{fig:zipkin_ui_detail}
\end{figure}
\subsection{Zipkin Data Model}
Zipkin requires data to be sent in JSON format. Requests to UI are sent as JSON arrays where the array elements are the spans. Zipkin understands the following fields of Span object:
\begin{itemize}
	\item \textbf{traceId} - unique id representing the complete trace. It can be either 128 or 64 bit long.
	\item \textbf{name} - human readable span name
	\item \textbf{id} - id of this span. At the current implementation, Zipkin UI supports span ids only to be 64-bit long.
	\item \textbf{parentId} - parent id of the current span.
	\item \textbf{timestamp} - the time when the span was created.
	\item \textbf{duration} - the duration of the span. It is the duration between the span creation and span closing.
	\item \textbf{annotations} - array containing standard Zipkin annotations. These annotations can be handled by user interface in specific way since the user interface understands the meaning of the content. The documentation specifies the following annotations:
	\begin{itemize}
		\item \textbf{cr} : timestamp of client receiving the span
		\item \textbf{cs} : timestamp of client sending the span
		\item \textbf{sr} : timestamp of server receiving the span
		\item \textbf{ss} : timestamp of server sending the span
		\item \textbf{ca} : client address
		\item \textbf{sa} : server address
	\end{itemize}
	\item \textbf{binaryAnnotations} - array of custom annotations. For example collected stack traces are sent as a binary annotation.
\end{itemize}

Except the \textit{annotations} and \textit{binaryAnnotations} fields, the fields are of simple string or number type. Annotations are objects with the three fields - annotation value, annotation name and the endpoint. Endpoint is another object specifying the address and port at the code where the span or particular annotation was recorded. Endpoints can also specify service name which may be used to search for particular spans.

Full example of data sent to Zipkin can be:
\begin{lstlisting}[emph={traceId, name, id, timestamp, duration, annotations, value, endpoint, serviceName, ipv4, port, binnaryAnnotations, key},emphstyle={\textbf}]
[
 {
    "traceId": "123456789abcdef",
    "name": "query",
    "id": "abcd1",
    "timestamp": 1458702548467000,
    "duration": 100743,
    "annotations": [
      {
        "timestamp": 1458702548467000,
        "value": "sr"
        "endpoint": {
          "serviceName": "example",
          "ipv4": "192.168.1.2",
          "port": 9411
        },
      }
    ],
    "binaryAnnotations": [
      {
        "key": "bytes_sent",
        "value": "1783"
        "endpoint": {
          "serviceName": "example",
          "ipv4": "192.168.1.2",
          "port": 9411
        },
      }
    ]
 }
]
\end{lstlisting}



\chapter{Implementation Details}
Mention interesting parts of the implementation
\section{Native Agent}
\subsection{Byte Class Parsing}
\subsection{Instrumentation}
\section{Instrumentation Server}
\subsection{Byte-Code Instrumentation}
\section{Zipkin Integration}
\subsubsection{Sending Data to Zipkin}

\chapter{Big Example}
\label{chap:big_example}

\chapter{Evaluation}
\label{chap:evaluation}
\section{Known Limitations}
here mention limitations with the instrumentation


\section{Platform demonstration}
\subsection{Deployment Strategies}
\subsubsection{Instrumentor per Application Node}
\subsubsection{Instrumentor per Whole Cluster}
\subsubsection{Optimizing the Deployment}
\subsection{Basic Building Blocks}
\subsection{Basic Demonstration}
\subsection{Optimizing the Solution}

grafy
\chapter{Conclusion}
\section{Comparison to Related Work}
\section{Future plans}
\subsection{Integration with well-known data collectors}
\subsection{Add support for Flame charts}

%%% Bibliography
\include{bibliography}
%%% Figures used in the thesis (consider if this is needed)
\listoffigures

%%% Tables used in the thesis (consider if this is needed)
%%% In mathematical theses, it could be better to move the list of tables to the beginning of the thesis.
%\listoftables

%%% Abbreviations used in the thesis, if any, including their explanation
%%% In mathematical theses, it could be better to move the list of abbreviations to the beginning of the thesis.


%%% Attachments to the master thesis, if any. Each attachment must be
%%% referred to at least once from the text of the thesis. Attachments
%%% are numbered.
%%%
%%% The printed version should preferably contain attachments, which can be
%%% read (additional tables and charts, supplementary text, examples of
%%% program output, etc.). The electronic version is more suited for attachments
%%% which will likely be used in an electronic form rather than read (program
%%% source code, data files, interactive charts, etc.). Electronic attachments
%%% should be uploaded to SIS and optionally also included in the thesis on a~CD/DVD.

\chapwithtoc{Attachments}
\begin{enumerate}
	\item Attachment 1: CD with the source codes of the Distrace tool.
	\item Attachment 2: List of examples available in Docker and instructions on how to run them.
	\item Attachment 3: Instructions on how to build and run
	\item Attachment 4: Native agent arguments
	\end{enumerate}
\openright

\setcounter{page}{1}
\chapter*{Attachment 2: List of Examples}
The source code of the thesis contains several examples. They can be run manually by first compiling the sources and then starting using the provided scripts. The examples are also available in the provided Docker machine. This docker machine contains all compile and run-time dependencies for the tool to be able to run and examples can be run in this machine as well.

The list of available examples:
\begin{enumerate}
	\item \textbf{DependencyInstrumentation} \newline
	This example just demonstrates the basic instrumentation functionality on dependent classes. It does not have any output do the user interface.
	\item \textbf{H2OSumMRTask} \newline
	This larger example demonstrates the tool on the H\textsubscript{2}O  machine learning platform. The cluster of three H\textsubscript{2}O  nodes is started and simple map-reduce tasks is executed within this cluster. The internals of map-reduce task are monitored and the associated spans are shown in the Zipkin user interface.
	\item \textbf{SingleJVMCallback} \newline
	This example demonstrates how the Distrace tool can be used to instrument and monitor the the callbacks. This example is using only a single JVM with multiple threads.
	\item \textbf{SingleJVMThread} \newline
	This example shows how thread can be instrumented in monitored. This example is running on a single JVM with multiple threads.
\end{enumerate}

\setcounter{page}{1}
\chapter*{Attachment 3: Building And Running}
To build both, the native agent and the core instrumentation server, the developer needs to obtain the Distrace tool sources\footnote{Distrace is available at \url{https://github.com/jakubhava/Distrace}.} and call \texttt{./gradlew build}\footnote{On Windows, \texttt{./gradlew.bat build}} command, which builds the tool and produces artifacts for the core instrumentation server and the native agent as well. The build process requires several dependencies to be available.

The instrumentation server requires Java 8 to be available. It has dependencies on several libraries, but these are downloaded automatically by the build process.

The native agent depends on several libraries which needs to be availalable on the system:
\begin{itemize}
	\item Boost-filesystem
	\item Boost-system
	\item Nanomsg
	\item Nanomsgxx
	\item Cmake
\end{itemize}
The last dependency is required to be able to build the native agent project. Any C++11 compliant compiler needs to be available as well.

Once the artifacts are build the user may extend the core instrumentation server with the instrumentation definition. The application can be further run using the following incantation:

\texttt{java -agentpath:"\$NATIVE\_AGENT\_LIB\_PATH=\$AGENT\_ARGS" -jar app.jar}

The \texttt{\$NATIVE\_AGENT\_LIB\_PATH} shell variable should point to the location of the native agent library and \texttt{\$AGENT\_ARGS} shell variable may contain any arguments passed to the native agent. The arguments are in the format \texttt{key=value} and are separated by the semicolon. Please see the Attachment 4 for the full list of available native agent arguments.

In order to be able to see the spans in the user interface, we need to start the Zipkin UI first. The user interface server may be started as: \texttt{java -jar zipkin.jar}\footnote{Zipkin Jar file may be downloaded at \url{https://github.com/openzipkin/zipkin} or is available at the attached CD.}.
\section{Running Docker Examples}
 In order to run this example without the need to set up all dependencies, the Docker container with this example is prepared. This container contains all Distrace dependencies and when started, is also automatically starts the Zipkin user interface on the default port. To run any example in Docker, Docker needs to be available, the projects source directory needs to be available as well\footnote{The sources are not actually needed for running the examples in docker, but the script which is used to start the docker machines is available there as well.} and the following script should be called: \texttt{./docker/run-test.sh}\footnote{On Windows, \texttt{./docker/run-test.cmd}}. 
 
Both these scripts expect a single argument representing the example name. When this argument is missing, the list of available examples is printed to the console.
\setcounter{page}{1}
\chapter*{Attachment 4 : Native Agent Arguments}
\setcounter{page}{1}
The native agent accepts several arguments which can be used to affect the agent behavior. In local instrumentation server mode, several arguments affect also the sever started from the agent. Available arguments are:
\begin{itemize}
	\item \textbf{instrumentor\_server\_jar} - specifies the path to the instrumentation server JAR. It is a mandatory argument in case the instrumentation server is supposed to run per each node of monitored application.
	\item \textbf{instrumentor\_server\_cp} - specifies the classpath for the instrumentation server. It can be used to add application specific classes on the server classpath which has the effect that the monitored application does not have to send to the server these classes if they need to be instrumented or if some class to be instrumented depends on them.
	\item \textbf{instrumentor\_main\_class} - specifies the main entry point for the instrumentation server. It is a required argument in case of local instrumentation server mode.
	\item \textbf{connection\_str} - specifies the type of connection between native agent and the instrumentation server. It is a mandatory argument in shared instrumentation server mode in which case the value is in format \texttt{tcp://ip:port} where ip:port is address of the instrumentation server. Otherwise, the agent and server communicates via inter-process communication and the argument can be set in format \texttt{ipc://identifier} where identifier specifies the name of pipe in case of Windows and name of the file used for IPC in case of Unix. However this value is set automatically at run-time if not explicitly specified as an argument.
	\item \textbf{log\_dir} - specifies the log directory for the agent and when running in local server mode, specifies the log directory for the server as well.
	\item \textbf{log\_level} - specifies the log level for the agent and when running in local server mode, specifies the log level for the server as well.
	\item \textbf{span\_exporter} - specifies the span exporter type. The value can be either \newline \texttt{directZipkin(ip:port)}, where \texttt{ip:port} is address of the Zipkin user interface or \texttt{disk(destination)}, where destination represents the output directory for the captured spans. 
	
	Custom span exporters are supported as well. In that case, the format of the value is fully qualified name of the span exporter with arguments in parenthesis, for example as \texttt{com.span.exporter(arguments)}
	\item \textbf{class\_output\_dir} - specifies the output directory for several helper classes received from the instrumentation server. This value is automatically set if not configured explicitly.
	\item \textbf{config\_file} - specifies path to a configuration file containing the agent configuration. It can contain all arguments mentioned above, except the configuration file argument. Argument entries in the configuration file are in the format \texttt{arg=value} and each entry is on a new line of the configuration file. 
\end{itemize}

\end{document}
