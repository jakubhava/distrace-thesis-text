\chapter{Implementation Details}
\label{chap:implementation}
This chapter explains several technical implementation details. It starts with explanation of the byte code parsing and instrumentation at the native agent part of the complete tool. The following section covers relevant parts of the instrumentation server which are the instrumentation together with creation of transformers and estimators. This chapter ends with a brief explanation of how the spans are exported to the Zipkin user interface and also, how the spans can be exported to a custom data format.

\section{Native Agent}
The native agent consists of several interesting technical parts. This section covers the instrumentation itself and also explains considered approaches during the development. The problem of instrumentation server requiring the dependencies for each instrumented class is explained together with the problem of instrumenting the classes with cyclic dependencies. The final solution is explain as well. 

In the following part, the internals of how JVM byte-code is parsed is explained.
\subsection{Instrumentation}
In general, the native agent does not perform the instrumentation but gets byte code for the the required class, sends the byte code to the instrumentation server and applies the instrumented byte code after receiving it from the client. 

The instrumentation server required all dependencies to be available for the currently instrumented class. This means that all other classes mentioned as part of method and field signatures, super classes or interfaces has to be available on the instrumentation server. To achieve this, two solutions have been tried but only the second solution shown to be feasible.

The first and unsuccessful solution was based on the fact that several on class file load hooks may be executed at several time in different threads. When the application loads a class, the class load hook event is triggered for it and its byte code is made available. In this method, the new class file load hook event was artificially enforced via the \texttt{RetransformClasses} method. This method accepts array of classes for which the hook should be re-thrown. In order to continue with the instrumentation of the original class, all dependent classes have to be instrumented in this solution first. This also means that in this approach, the classes with cyclic dependencies are not supported. In order to instrumented such a class, all dependencies have to be instrumented first which is also the class itself.

This solution had also different problem. Since a number of dependencies can be significant, the problem of too many threads being opened at a single time has also appeared. 

The second and currently used solution is based on the fact that the Java class files may be accessed as a resource using the class loader of the class currently being load. Disadvantage of this solution that the developer may override the \texttt{getResourceAsStream} method on the custom class loader and not provide access to he class files. This is a limitation of the thesis. However, when a such event happens, the instrumentation does not end with the exception but first, the attempt to load the class using a different class loader is done. 

In this solution, the instrumentation server is first asked whether the current class should be instrumented based on the server's extensions. If the class is designed to be instrumented, its byte code is send to the instrumentation server (only in case if the byte code for the class is not already available). Then, dependencies are scanned via parsing the raw JVM byte code which is explained in detail in the following section. Dependency loading is recursively called for each new dependent class until the class does not have any other dependencies or if all the dependencies are already uploaded to the instrumentation server. Once all dependencies for a class have been sent to the server, the instrumentation is invoked and the agent waits for the new byte code. 

 Also several helper classes are sent back to the native agent at the stage where the class is checked whether it should be instrumented or not. The classes are auxiliary Byte Buddy classes and also instances of \texttt{LoadedTypeInitializer} class. The initializers are sent as serialized 
instances and therefore their defining class has to be available in the application. This is achieved in the agent initialization phase where several required classes are sent to the application from the server. The instances are saved to a map which maps the initializer name to its serialized representation. This initializer is later using during the class preparation phase to set the static interceptor field of the instrumented class as mentioned in the previous sections.

The auxiliary classes are classes created at run-time during the instrumentation on the server and have to be available on the applications machine as well. This is achieved by loading the byte code for the auxiliary class, saving the the class as a java class file on the disk and making it available by adding the class on the application's class path.

\subsection{Byte Code Parsing}
explain byte code structure

\section{Instrumentation Server}
\subsection{Byte-Code Instrumentation}
\subsection{Transformers and Estimators}
\section{Zipkin Integration}
\subsubsection{Sending Data to Zipkin}
