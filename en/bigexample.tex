\chapter{Big Example}
\label{chap:big_example}
This chapter demonstrates how Distrace can be used on bigger example and also serves as the user manual for creating custom monitoring applications. We will show how all steps from creating the application up to running the application and seeing the observed results.

This example is based on the H\textsubscript{2}O\footnote{More information about H\textsubscript{2}O can be found on \url{https://www.h2o.ai} and \url{https://github.com/h2oai}} open-source fast scalable machine learning platform. This platform supports various methods for building machine learning models, methods such as deep learning, gradient boosting or random forests. The core of the tool is written in Java and clients for different languages exist as well. Internally, H2O is using map-reduce computation paradigm to perform various tasks across the cluster.

The goal of this example is to monitor subset of map-reduce tasks and see their visualizations. This can help reasoning about performance of the platform and can discover unwanted delays in computations. This chapter first describes relevant parts of the H2O platform in more details. Then in the following sections we describe in steps how to extend the core instrumentation library for H2O purposes. Lastly, we show how this example can be started and how visual output can be interpreted. 
This full example is also available in the attached source code of the thesis.

\section{H2O In More Details}

\section{Building the Core Server and Native Agent}

\section{Extending the Core Instrumentation Server}

\section{Configuring and Running the Application}

\section{Interpreting the Results}