\chapter{Design}
\label{chap:design}
\section{Basic Concepts}
\subsection{Spans}
\label{subsec:spans}
\section{Native Agent}
mention here the issue with running more JVMs inside one process
\subsection{Structure Overview}
\subsection{Instrumentation}
mention issues with circular dependencies but leave how it is implemented into the next chapter
\subsection{Instrumentation API}
\subsection{Native Agent Arguments}
\subsection{Used JVMTI Callbacks}
\section{Instrumentation Server}
 There is also deployment strategy which allow the programmer to build the server together with the application classes. This has the effect that when a class needs to be instrumented, the agent is not required to send it to the server because server already has the class available. The solu
\subsection{Instrumentation Protocol}
\subsection{Communication Modes}
\subsection{Class Caching}
\subsection{Custom Service Loader}
\subsection{Public interfaces}
\subsection{Extending the Server}
.. instrumentation server can run on the same node or over the network. Instrumentation server can have client code attached or not.
\subsection{Class Loaders}
\subsection{JSON Generation}
\subsection{Deployment Mode}
\label{sec:deploy_mode}
\section{User Interface}
\label{sec:zipkin_ui}
\subsection{Zipkin UI Overview}
\subsection{Zipkin Data Model}
\subsection{Zipkin JSON Format}
\section{Collectors}
Should I mention the collectors ? It may be sufficient to have send data right to zipkin for demonstration purposes
